%% inicio, la clase del documento es iccmemoria.cls
\documentclass{iccmemoria}
\usepackage{amsmath}
\usepackage{algorithm}
\usepackage{tabularx}
\usepackage{placeins}
\usepackage[noend]{algpseudocode}

\algdef{SE}[DOWHILE]{Do}{doWhile}{\algorithmicdo}[1]{\algorithmicwhile\ #1}%
\newcommand{\abs}[1]{\lvert#1\rvert}
\newcommand{\norm}[1]{\lVert#1\rVert}

%% datos generales y para la tapa
\titulo{Análisis experimental de IIQS para extender el soporte a secuencias de elementos repetidos}
\tituloEng{Experimental analysis of (I)IQS to fine-tune support for arrays with repeated elements}
\author{Erik Andrés Regla Torres}
\supervisor{Rodrigo Andrés Paredes Moraleda}
\informantes
	{NN 1}
	{NN 2}
%\adicional{(sólo por si se necesita agregar algún otro profesor)}
%\director{Profesor del ramo Memoria de Título} % no entiendo para que esta esto :-(
%innocuous comment :-p
%cambio 2
\date{mes, año}

%% inicio de documento
\begin{document}

%% crea la tapa
\maketitle
%cambio 5
%cambio 3
%cambio 4

%% dedicatoria
\begin{dedicatory}
  Dedicated to... someone ?
\end{dedicatory}

%% agradecimientos
\begin{acknowledgment}
  Agradecimientos a ... (how the fuck do I choose whom to acknowledge?)
\end{acknowledgment}

%% indices
\tableofcontents
\listoffigures
\listoftables

%% resumen
\begin{resumen}
  I'm gonna write the summary as the last part.
\end{resumen}

%% abstract

%% contenido del primer capítulo
%% This should be the last thing to write
\chapter{Introduction}
Aquí va el texto del capítulo 1...

\section{Context}
Aquí va el texto de la primera sección del capítulo 1...

\section{Application areas}
Aquí va el texto de la primera subsección de la primera sección del capítulo 1...

\section{Problem description}
Aquí va el texto de la segunda subsección de la primera sección del capítulo 1...

\section{Goals}
Aquí va el texto de la segunda subsección de la primera sección del capítulo 1...

\subsection{General goals}
Aquí va el texto de la segunda subsección de la primera sección del capítulo 1...

\subsection{Specific goals}
Aquí va el texto de la segunda subsección de la primera sección del capítulo 1...

\section{Document Structure}
Aquí va el texto de la segunda subsección de la primera sección del capítulo 1...


% \section{Hiearchies of algorithm design}
% \section{Experimental process}

\chapter{Methodology}
\chapterimage[height=3cm]{./fragments/memes/thor_tools}
\label{CHAPTER:METHODLOGY}

In this chapter we revisit the methodology explained in Section~\ref{SECTION:EXPERIMENTAL_ALGORITHMICS} in order to understand better how it is applied to our problem at hand.


\section{Rationale}
IncrementalQuickSort and its introspective version, IntrospectiveQuickSort, already have their theoretical analyses for the worst case instances. But such theoretical analysis is not always feasible, sometimes not easy and most of the time it is not realistic. As a practical example of it, when testing IQS against HeapSort for a full array sorting under architectures with small cache memory, IQS outperforms vastly HeapSort as it trashes the cache on each iteration. But when cache units are large enough to support the entire array, there is no point on using IQS, as most operations are actually solved on cache directly. There is a huge gap when it comes to practice on algorithm design, and IIQS is also not free of such problems.\\

The main issue arose when it comes to the analysis of the median-of-medians effects on the partition. As the execution of this algorithm offsets itself on each partition, it is the equivalent to run continuously a process which reduces the overall disorder of the sequence\footnote{We do not talk explicitly of any disorder metric seen in Section~\ref{SEC:MEASURING_DISORDER} as the effect depends on the process behind each execution. In this sense, we want our definition to be abstract and not to be tied with any algorithm implementation.}.\\

This effect displaces the elements on the sequence towards their expected position on it, due to the adaptive-sorting nature of both IQS and IIQS makes the overall running time dependent on the element distribution. This makes really hard the use of standard techniques like amortized analysis to study the behavior of IIQS. Even worse, due to the increased complexity of the algorithm, its theoretical analysis is likely to differ from the practical results.\\

The problem at hand is to modify the current implementation of IQS and IIQS to support an extra case, which is when the sequence of elements can have repeated elements on it. This is now a problem for many reasons as it messes up the pivot selection heuristics and partition stages of the original algorithm.\\

Due to the aforementioned reasons, we want to take an experimental approach to analyze this new instance and use the results to guide the development of an extension of this algorithm.\\

\FloatBarrier
\section{Methodology foundationals}
\label{SECTION:METHODOLOGY_FOUNDATIONALS}

\subsection{Algorithm instantiation hierarchy}
The main goal of this work is to devise if we can design version of both IQS and IIQS which can avoid the worst case when dealing with sequences that hold repeated elements. Once this goal is accomplished, then we need to study its behaviour in order to check ways to deliver the same performance for both repeating and non repeating sequences. Thus our instantiation hierarchy is as follows:\\

\subsubsection{Metaheuristics and algorithm paradigms}
IncrementalQuickSelect and IntrospectiveIncrementalQuickSelect are algorithms used for partial sorting, belonging to the \textit{partition-based adaptive sorting} family which are our optimization target using repeated elements on a sequence.\\

\subsubsection{Algorithms}
As both IQS and IIQS are partition-based algorithms, they both share common elements and routines, namely:\\

\begin{itemize}
    \item \textbf{Next}:This is the main process on both IQS and IIQS. It performs a minima extraction. It expected average running time is $O(n + log_2{n})$ on which $n$ is the size of the sequence being passed as input.
    \item \textbf{Partition}: Partitions a given sequence into three subsequences $p_1$, $p_2$ and $p_3$ which follows that $\forall~p_i \in p_1,~\forall~p_j \in p_2:~ p_i < p_j$ and $\forall~p_i \in p_3,~\forall~p_j \in p_2:~p_i < p_j$. It expected average running time is $O(m)$ on which $m$ is the size of the sequence being passed as input following $m \leq n$ on which $n$ is the total sequence length.
    \item \textbf{Swap}: Swaps two elements in the sequence in-place. It expected average running time is $O(1)$.
    \item \textbf{PushStack}: Pushes an element into the stack. It expected average running time is $O(1)$.
    \item \textbf{PullStack}: Pulls an element from the stack. It expected average running time is $O(1)$.
\end{itemize}

As for IIQS exclusive use routines we can mention:
\begin{itemize}
    \item \textbf{BFPRT:} Implementation of median of medians algorithm~\cite{Blum_Floyd_Pratt_Rivest_Tarjan_1973}. This algorithm is used as a fallback option for the random selection pivot selection performed during each iteration of IQS. It expected average running time is $O(m)$ on which $m \leq n$ is the size of the sequence being passed as input.
    \item \textbf{Median:} Sorts in-place an array of fixed size\footnote{Whilst on literature a median of medians of five elements is suggested, we do not want to tie the implementation of the algorithm to a fixed value, but rather become this size a parameter of our algorithm.} and then retrieves the element in the middle position. It expected average running time is $O(1)$, despite the complexity of the sorting mechanism used as the time used is constant and is not in function of the sequence length.
\end{itemize}

\subsubsection{Source program}
As for the implementation, our language of choice was C++ in conjuntion with Boost libraries for argument parsing. \\

\subsubsection{Object code}
Object code is obtained via direct compilation of the main source file. Due to portability concerns, this process is triggered via Makefiles but no special or separate treatment is done for the compilation artifacts.\\

\subsubsection{Process}
All experiments are executed on userspace without any extra execution privileges.\\

\subsection{Algorithm design hierarchy}

\subsubsection{System structure}
Experiments are structured in a way that prioritizes execution flexibility over convention. Initial versions of the implemented algorithm were developed using C, but this language as is has some problems when it comes to understanding from the developer's standpoint. In this aspect, C++ (which can be seen as an class-oriented version of C) helps to perform an easier modularizaton of the code, while maintaining the macro flexibility of C. Apart from the main algorithm implementations, we can identify the following structures:\\

\subsubsection{Main driver}
The entrypoint of our compilation unit is our daily driver, which takes charge of argument parsing, main execution and high-level operations as snapshot dumping.  Runtime options are parsed with the help of Boost's \texttt{boost::program\_options} library, which takes charge of parsing and converting arguments to their corresponding types. \\

The list of available arguments to pass to the main driver are shown in the table shown in Table~\ref{TABLE:ARGUMENTS}:\\


\begin{table}[!ht]
    \centering

    \begin{tabularx}{\linewidth}{|r|r|X|}
        \hline
        STD type & CLI option & Description \\
        \hline
        \texttt{bool} & \texttt{--log\_pivot\_time} & Enables clock for pivot selection runtime\\
        \hline
        \texttt{bool} & \texttt{--log\_iteration\_time} & Enables clock for iteration runtime \\
        \hline
        \texttt{bool} & \texttt{--log\_extraction\_time} & Enables clock for element extraction runtime\\
        \hline
        \texttt{bool} & \texttt{--log\_swaps} & Enables logging of swap information \\
        \hline
        \texttt{bool} & \texttt{--use\_bfprt} & Enables median of medians instead of random selection (only for IIQS) \\
        \hline
        \texttt{bool} & \texttt{--use\_iiqs} & Enables use of IIQS instead of IQS \\
        \hline
        \texttt{bool} & \texttt{--use\_random\_pivot} & Enables random pivot seletion \\
        \hline
        \texttt{bool} & \texttt{--enable\_reuse} & Enables pivot reuse \\
        \hline
        \texttt{double} & \texttt{--alpha\_value} & Sets $\alpha$ value \\
        \hline
        \texttt{double} & \texttt{--beta\_value} & Sets $\beta$ value \\
        \hline
        \texttt{double} & \texttt{--pivot\_bias} & Sets pivot selection bias \\
        \hline
        \texttt{int} & \texttt{--random\_seed\_value} & Sets random seed value \\
        \hline
        \texttt{std::size\_t} & \texttt{--input\_size} & Experiment input size \\
        \hline
        \texttt{std::size\_t} & \texttt{--extractions} & Experiment extractions to perform \\
        \hline
        \texttt{std::string} & \texttt{--input\_file\_value} & Input file path \\
        \hline
        \texttt{std::string} & \texttt{--output\_file\_value} & Output file path \\
        \hline    
    \end{tabularx}
    \caption{Arguments for main driver program}
    \label{TABLE:ARGUMENTS}
\end{table}

\subsubsection{Snapshot spec}
The information about the program execution is stored on \textit{snapshots}, which are inspired on Valgrind profiler stats. Snapshot contains a rather large amount of information, from program configurations, options, and current state. In order to capture snapshots, a set of precompiler macros has been defined to enable timing of whole routines, partial sections, and conditional values. \\

In order to minimize the performance hit, there is only one instance of \texttt{snapshot\_t} being kept on memory at all times which is passed as reference to the instances of IQS and IIQS. This snapshot is only saved on the record array log on RAM at certain keypoints. This record array log is preinitialized before the execution in order to avoid allocation calls during the execution phase. Currently the time unit used for benchmarking is \texttt{std::chrono::nanoseconds}, defined on the \texttt{TIME\_UNIT} macro.\\

The table on Table~\ref{TABLE:SNAPSHOT_STRUCTURE} shows the logged metrics described on the \texttt{snapshot\_t} structure:\\

\begin{table}[!ht]
    \centering
    \begin{tabularx}{\linewidth}{|r|X|}%|X|}
    \hline
    STD type & logged variable \\ % & Description \\ %& \\
    \hline
    \texttt{TIME\_UNIT} & \texttt{iteration\_time} \\ %& \\
    \hline
    \texttt{TIME\_UNIT} & \texttt{total\_iteration\_time} \\ %& \\
    \hline
    \texttt{TIME\_UNIT} & \texttt{partition\_time} \\ %& \\
    \hline
    \texttt{TIME\_UNIT} & \texttt{total\_partition\_time} \\ %& \\
    \hline
    \texttt{TIME\_UNIT} & \texttt{bfprt\_partition\_time} \\ %& \\
    \hline
    \texttt{TIME\_UNIT} & \texttt{total\_bfprt\_partition\_time} \\ %& \\
    \hline
    \texttt{TIME\_UNIT} & \texttt{extraction\_time} \\ %& \\
    \hline
    \texttt{TIME\_UNIT} & \texttt{total\_extraction\_time} \\ %& \\
    \hline
    \texttt{size\_t} & \texttt{current\_extraction\_executed\_partitions} \\ %& \\
    \hline
    \texttt{size\_t} & \texttt{total\_executed\_partitions} \\ %& \\
    \hline
    \texttt{size\_t} & \texttt{current\_iteration\_partition\_swaps} \\ %& \\
    \hline
    \texttt{size\_t} & \texttt{total\_executed\_partition\_swaps} \\ %& \\
    \hline
    \texttt{double} & \texttt{current\_iteration\_longest\_partition\_swap} \\ %& \\
    \hline
    \texttt{double} & \texttt{total\_executed\_longest\_partition\_swap} \\ %& \\
    \hline
    \texttt{size\_t} & \texttt{current\_iteration\_executed\_bfprt\_partitions} \\ %& \\
    \hline
    \texttt{size\_t} & \texttt{total\_executed\_bfprt\_partitions} \\ %& \\
    \hline
    \texttt{size\_t} & \texttt{current\_iteration\_bfprt\_partition\_swaps} \\ %& \\
    \hline
    \texttt{size\_t} & \texttt{total\_executed\_bfprt\_partition\_swaps} \\ %& \\
    \hline
    \texttt{double} & \texttt{current\_iteration\_longest\_bfprt\_partition\_swap} \\ %& \\
    \hline
    \texttt{double} & \texttt{total\_executed\_longest\_bfprt\_partition\_swap} \\ %& \\
    \hline
    \texttt{double} & \texttt{current\_extracted\_pivot} \\ %& \\
    \hline
    \texttt{size\_t} & \texttt{current\_stack\_size} \\ %& \\
    \hline
    \texttt{size\_t} & \texttt{total\_pushed\_pivots} \\ %& \\
    \hline
    \texttt{size\_t} & \texttt{total\_pulled\_pivots} \\ %& \\
    \hline
    \texttt{size\_t} & \texttt{current\_iteration\_pushed\_pivots} \\ %& \\
    \hline
    \texttt{size\_t} & \texttt{current\_iteration\_pulled\_pivots} \\ %& \\
    \hline
    \texttt{size\_t} & \texttt{current\_extraction} \\ %& \\
    \hline
    \texttt{size\_t} & \texttt{input\_size} \\ %& \\
    \hline
    \texttt{char} & \texttt{snapshot\_point} \\ %& \\
    \hline
\end{tabularx}
\caption{Snapshot structure}
\label{TABLE:SNAPSHOT_STRUCTURE}
\end{table}


\subsubsection{Algorithm and data structure design}
Testbench implementation is divided into four major components:\\

\begin{itemize}
    \item{
        \textbf{IQS C++ Implementation}: Base C++ implementation of IQS with support for C++ STD container classes. One of the differences with the standard implementation of IQS is the presence of \texttt{IQS::random\_between} and \texttt{IQS::biased\_between} methods, which allows control over the pivot selection methods.
    }
    \item{
        \textbf{IIQS C++ Implementation}: Base C++ implementation of IIQS with support for C++ STD container classes. Inherits all components for IQS so this implementation only overloads \texttt{IQS::next} method and adds \texttt{IIQS::bfprt}, intended to support the extra operations needed by IIQS.
    }
    \item{
        \textbf{IQS low-level C++ Implementation}: C++ implementation of IQS without support for C++ STD container classes, relying only on direct memory allocation. This implementation was not benchmarked as it is only intended to be used as reference.
    }
    \item{
        \textbf{IIQS low-level C++ Implementation}: C++ implementation of IIQS without support for C++ STD container classes, relying only on direct memory allocation. This implementation was not benchmarked as it is only intended to be used as reference. All methods from low-level IQS are inherited here.
    }
\end{itemize}

\subsubsection{Implementation and algorithm tuning}
On the original IIQS analysis, randomized sequences and sorted sequences were used as tests. The original problem constrained all worst case input instances to be ordered sequencies in order to ease understanding. On ordered sequences is easier to see when a pivot selection fails by misusing the stack, thus not reducing the problem size. But since we now are dealing with repeated elements, new sequences for input are needed to test such cases.\\

\begin{itemize}
        
    \item{\textbf{Randomized sequences}: 
    This is our classical test case, on which all the elements are shuffled without any special criteria.}

    \item{\textbf{Ascending sequences}: 
    Used to generate a synthetic worst-case instance for IQS, this sequence is ordered in ascending order.}

    \item{\textbf{Descending sequences}: 
    Used to generate a synthetic worst-case instance for IQS, this sequence is ordered in descending order.}

    \item{\textbf{Constrained classes}: 
    Given $m < n$ the number of classes on the sequence, we want to test the effect of the ratio $\frac{m}{n}$ for a fixed number classes to devise if there is a relationship between the number of classes and the running time of the algorithm. This input is shuffled after its generation.}

    \item{\textbf{Constrained classes with random noise}: 
    In addition to the previous instance, we also add a random number of elements which do not belong to any instances of $m$ to induce random noise on the sample. This input is shuffled after its generation.}

    \item{\textbf{Shuffled sequences with sorted segments}: 
    Based on a mix of \textit{Runs}, \textit{SUS} and \textit{SMS.SUS} metrics for adaptive sorting, this input attempts to test the effect of presortedness on the execution of IQS and IIQS. To generate this input we first generate a shuffled input and then for each subsegment of the shuffled sequence we execute a partial sorting.}

    \item{\textbf{Randomized sequence with ignored noise}: 
    This input is intended to test if discontinuities on the sorting process can affect the performance by ignoring certain swaps. To accomplish this, we take a randomized sequence and then for a given amount of elements on the sequence we put the value that belong to their position.}
\end{itemize}

Aditionally, due to the nature of the problem, we have decided to constrain the following two aspects of the implementation in order to ensure performance and replication of results. So, they can be peer-validated at a later stage. Replication is achieved by taking a monte-carlo simulation~\cite{10.5555/1614191} approach using the following means:\\

\begin{itemize}
    \item{\textbf{Fixed seeding}
    For all experiments, all inputs are generated beforehand on the same instance of the machine in sequential order and providing the same seed for all random number generators.}
    
    \item{\textbf{Systematic randomization}
    For all processes that require randomization, the random values provided are extracted from a separate file beforehand, this ensures that all extractions of random numbers for the use of the algorithm are delivered in the same order across executions.}
\end{itemize}

\subsubsection{Code tuning}
%only one instance
%preinstanciation before clocking
\begin{itemize}
    \item{\textbf{Unique snapshot intance}: 
    In order to minmize memory consumption and allocation operations only one snapshot instance is initialized for a whole experiment, and it is passed as reference along the whole program.}
    \item{\textbf{Memory pre-allocation}: 
    All test cases, files, snapshot space, and random generated number are computed by an external process and they are fed into the program via file inputs which are read using STL \texttt{std::ifstream} and initialized before all tests start, so memory is allocated already at this point in order to prevent reallocation operations.}
    \item{\textbf{Unique source of truth}:
    All random numbers used along implementations are extracted from a unique source, from the same allocated space during runtime. All alocations are performed before the main execution and clocks start running in order to ensure that inicialization process does not affect runtimes due to memory allocation overhead.}
\end{itemize}

\subsubsection{System software}
Our compiler is GNU GCC 9.3.0 configured for a x86\_64-linux-gnu target with posix thread modeling. All compiler optimizations are disabled in order to track time more acurrately.
\subsubsection{Platform and hardware}
The specs of the machine used are shown in the table on Table~\ref{TABLE:SPECS}:\\

\begin{table}[!ht]
    \centering
    \begin{tabularx}{\linewidth}{|l|X|}
        \hline
        Item & Product ID \\
        \hline
        Processor  & Ryzen 5 Series 3600, 6C/12T 3.6 GHz base processor clock 4.2 GHz max boost, 35MB cache, unlocked clock settings \\
        \hline
        Memory  & Team T-FORCE DARK Za 32GB (2 x 16GB) DDR4 3600 MHz (PC4 28800) TDZAD432G3600HC18JDC01, dual-channel enabled, XMP profile 1 enabled\\
        \hline
        Storage  & WD M.2 SSD 480GB WDS480G2G0B \\
        \hline
        Motherboard  & MSI B450M PRO-VDH MAX, AM4 \\
        \hline
        Power Suppy Units  & EVGA 600W W1, 80+ Certified\\
        \hline
        Video Adapter  &  Galax Video NVIDIA GeForce GTX1650 1-Click OC  \\
        \hline
        Operating System  &  Pop!\_OS 20.04 LTS  \\
        \hline
        Kernel  &  Linux rspk-shoukugun 5.4.0-7629-generic \#33\~{}1589834512\~{}20.04\~{}ff6e79e-Ubuntu SMP Mon May 18 23:29:32 UTC  x86\_64 x86\_64 x86\_64 GNU/Linux\\
        \hline
        Linux Version  & Linux version 5.4.0-7629-generic (buildd@lcy01-amd64-013) (gcc version 9.3.0 (Ubuntu 9.3.0-10ubuntu2)) \#33\~{}1589834512\~{}20.04\~{}ff6e79e-Ubuntu SMP Mon May 18 23:29:32 UTC \\ 
        \hline
    \end{tabularx}    
    \caption{Test machine specs}
    \label{TABLE:SPECS}
\end{table}
\FloatBarrier
\section{Experimental process breakdown}
% first generate inputs
% perform gridsearch
% analyze results
% present
% raise next question
Now, we present a breakdown of the experimental process definitions needed in order to begin executing our experiments.\\

\subsection{Experimental cycle}
As mentioned before in Section~\ref{SUBSECTION:EXPERIMENTAL_ALGORITMICS_CONSIDERATIONS}, this process is being handled in a cyclic manner, and each experiment is controlled by their own Jupyter Notebook. Now we describe the steps followed through the realization of the experiments of this report.\\

\subsubsection{Hypothesis}
We begin by raising a question and a doable answer to it which we want to prove if it holds for the current experimentation cycle. In the context of this report, hypothesis are code tuning improvements that we want to check before building a solution for our problem.\\

\subsubsection{Input and execution setup}
After defining our topic, we start generating inputs and planning the experiment execution. As for the inputs, we already defined in Section~\ref{SECTION:METHODOLOGY_FOUNDATIONALS} how the inputs are be generated, as for each experiment inputs are to be selected accordingly on what we want to test or explore. \\

On the other hand, as the main driver already accepts a defined set of parameters to control the execution, those are used to setup the experiment environment. From now on, we refer to the cross product of the combination of inputs and program parameters as the \textit{search space}.\\

\subsubsection{Execution}
At this point we execute a GridSearch over out \textit{search space} in sequential order.This way we reduce the amount of disturbances that the experiment can suffer. After the execution of the experiments, snapshots are stored on a single ASCII file using a comma separated schema, which is is later used for the analysis.\\

\subsubsection{Analysis}
The results are gathered on Jupyter and examinated in order to get insight on the phenomena and to check if the hypothesis is valid or not. After we gather enough information, a discussion on the results is held, which is written together with the results on this report.\\

\subsection{Metrics and indicators}
In order to evaluate the experiments, we need to define beforehand some metrics to determine which aspects of the execution are evaluated during the experiments. The difference between comparing raw data and use metrics is the pre-processing being made in order to gain useful insight about what is going on under the hood before executing our first benchmark.\\

\subsubsection{Swaps}
Using the definitions for measuring disorder given in Section \ref{SEC:MEASURING_DISORDER}, we can establish metric for complete sorting algorithms. Both IQS and IIQS falls into the definition of incremental sorting as shown on Section~\ref{SEC:INCREMENTAL_SORTING}, making its use natural to us. But as our study consists on analysing the behaviour of extractions and not of a complete sorting execution, such metrics are unrealistic to such purposes and unnatural.\\

Still, such definitions can help us to establish a base to define our own disorder metric given the following known premises:\\

\begin{itemize}
    \item Both IQS and IIQS rely on \textit{partition} in order to perform the partial sorting the same way as \textit{QuickSort}.
    \item QuickSort, being a adaptive sorting algorithm, is influenced by presortedness.
    \item \textit{Dis}, \textit{Max}, \textit{Exc}, \textit{Rem}, \textit{Runs}, \textit{SUS}, and \textit{SMS.SUS} are applicable metrics for \textit{QuickSort}.
    \item Both IQS and IIQS perform their heavy lifting at the partition stage.
\end{itemize}

Then, the minimum common denominator of the aforementioned premises is the behaviour of swap operations. Given the nature of \textit{partition} operation, it is expected to not exchange any elements (\textit{Exc}) on a sorted sequence, and transitively, each segment of every iteration of IQS must follow the same property given that it is being performed in-place.\\

On the other hand, as it can be seen on the executions for worst case of IQS, as there are long ascending sub-sequences on the input (\textit{Runs}, \textit{SUS}, \textit{SMS.SUS}), the fastest the partition stage ends, as it is not performing any swaps and in some cases not storing any pivots at all.\\

Now when a swap operation occurs at the partition stage, the elements being swapped and the pivot share a partial sorting relationship between them. Given that this relationship can occur at any point of the sequence we can state the following aspects of the swap operation:\\

\begin{itemize}
    \item The longest the distance of the swap being performed from the pivot, the elements are more far away from their actual position (\textit{Dis}, \textit{Max}).
    \item A sorted sequence does not perform any swaps on their partition stage.
    \item Given that all elements are being sorted in-place, and partitions are executed in a recursive way respect the pivot position, the previous two properties are transitive respect the partitions that are being generated during the extraction of a minima.
\end{itemize}

Then, we establish the following two disorder metrics for IQS and IIQS:\\

\begin{itemize}
    \item \textbf{N\_SWAPS}: As the number of swaps performed for a given iteration of IQS. This metric can be extended as cumulative regarding the extraction of a minima. The values for \textbf{N\_SWAPS} are in the range $[0, n^2]$ for a given iteration.
    \item \textbf{MAX\_SWAP}: As the longest swap performed during a iteration of IQS. This metric only applies to each individual execution of partition, as the maximum distance is bound to decrease on each iteration. The values for \textbf{MAX\_SWAP} are in the range $[0, m]$, on which $m$ is the size of the current partition which follows $m \leq n$.
\end{itemize}

\subsubsection{Stack operations}
A key aspect of IQS and IIQS is the amount of extra memory needed to perform the sorting. Under normal conditions, IQS requires $log_2(n)$ extra space in order to maintain all pivots for the first extraction. This makes the first extraction the most time-consuming operation as it partitions over $n$ elements and pushes $log_2(n)$ pivots into the stack in average.\\

One of the problems of IQS was the fact that in certain cases, $n$ pivots get stored on the stack, forcing for the non-introspective version to fix the stack size at $n$, whilst the introspective version due to its most stable behaviour can be safely set at $log_{1.7}(n)$ thanks to the median-of-median algorithm effect.\\

In light of the aforementioned facts, it makes sense that any modification being made to IQS or IIQS must also compare the performance of the stack growth due to caching and memory consumption concerns. In contrast to the previous work on IIQS, this version of the analysis also considers other extra metrics such as number of pulled elements and number of pushed elements per iteration and per minima extraction, as their behaviour is directly connected to the time used by the partitioning stage.\\

\subsubsection{Number of executed subroutines}
In line with the stack problem, it is sane to establish if the number of elements in the stack is in direct relation to the executions of the partition routine, and in the same way, the executions of the partition routine has direct relationship with the number of elements extracted. In this regard, we also want to log the time that both routines get called during minima extraction to study if they had any effect on the total running time of IQS and IIQS.\\

\subsubsection{Pivot bias}
The preferred way of testing worst-case executions on partition-based sorting algorithms is to fix the pivot selection to a position which ensures the worst outcome each time. This is the main reason on why introduction of randomization for selection for pivots is so effective on maintaining expected average case on such algorithms. For IQS, worst case comes from choosing the lowest or the highest element on the sequence which can be accomplished by fixing the pivot position on the edges of the sequence to be partitioned, given that the entire sequence is already sorted.\\

But when testing synthetic , it is not proven if there is a position which performs better for certain cases than selecting the middle element as pivot on partition based algorithms (given the same constraints as the previous paragraph), nor if this bias for pivot selection can be used to tune the algorithm beforehand for certain cases.\\

\subsubsection{Clocked routines}
Not all parts of the program are subjected to monitoring via snapshots, as this approach is both nonsensical and non practical. We just change the execution of the program on certain points of the execution in order to gather metrics or to push elements to the log array. \\

Currently defined sections to be used as snapshot points are shown in the table on Table~\ref{TABLE:POINTS}:

\begin{table}[!ht]
    \centering
    \begin{tabularx}{\linewidth}{|r|r|X|}
        \hline
        Program flag & code & Description \\
        \hline
        \texttt{EXTRACTION\_STAGE\_BEGIN} & \texttt{10} & Start of the minima extraction\\
        \hline
        \texttt{EXTRACTION\_STAGE\_END} & \texttt{20} & End of the minima extraction \\
        \hline
        \texttt{ITERATION\_STAGE\_BEGIN} & \texttt{30} & Begin of a IQS or IIQS iteration \\
        \hline
        \texttt{ITERATION\_STAGE\_LOOP} & \texttt{40} & Middle point of a IQS or IIQS iteration \\
        \hline
        \texttt{ITERATION\_STAGE\_INTROSPECT} & \texttt{41} & Introspect stage of IIQS \\
        \hline
        \texttt{ITERATION\_STAGE\_END} & \texttt{50} & End of IQS or IIQS iteration \\
        \hline
        \texttt{PARTITION\_STAGE\_BEGIN} & \texttt{60} & Start of partitioning stage \\
        \hline
        \texttt{PARTITION\_STAGE\_END} & \texttt{70} & End of partitioning stage \\
        \hline
    \end{tabularx}
    
    \caption{Timed sections}
    \label{TABLE:POINTS}
\end{table}

\FloatBarrier



% \subsection{Pilot experiments}
% \subsubsection{Incremental version of BFPRT}
% \subsubsection{Introspective step rule changes}
% \subsubsection{Three-way partition pivot location bias}
% \subsubsection{Three-way partition pivot store}
% \subsubsection{Change rules to store pivots}


% \section{Hiearchies of algorithm design}
% \section{Experimental process}

\chapter{Methodology}
\chapterimage[height=3cm]{./fragments/memes/thor_tools}
\label{CHAPTER:METHODLOGY}

In this chapter we revisit the methodology explained in Section~\ref{SECTION:EXPERIMENTAL_ALGORITHMICS} in order to understand better how it is applied to our problem at hand.


\section{Rationale}
IncrementalQuickSort and its introspective version, IntrospectiveQuickSort, already have their theoretical analyses for the worst case instances. But such theoretical analysis is not always feasible, sometimes not easy and most of the time it is not realistic. As a practical example of it, when testing IQS against HeapSort for a full array sorting under architectures with small cache memory, IQS outperforms vastly HeapSort as it trashes the cache on each iteration. But when cache units are large enough to support the entire array, there is no point on using IQS, as most operations are actually solved on cache directly. There is a huge gap when it comes to practice on algorithm design, and IIQS is also not free of such problems.\\

The main issue arose when it comes to the analysis of the median-of-medians effects on the partition. As the execution of this algorithm offsets itself on each partition, it is the equivalent to run continuously a process which reduces the overall disorder of the sequence\footnote{We do not talk explicitly of any disorder metric seen in Section~\ref{SEC:MEASURING_DISORDER} as the effect depends on the process behind each execution. In this sense, we want our definition to be abstract and not to be tied with any algorithm implementation.}.\\

This effect displaces the elements on the sequence towards their expected position on it, due to the adaptive-sorting nature of both IQS and IIQS makes the overall running time dependent on the element distribution. This makes really hard the use of standard techniques like amortized analysis to study the behavior of IIQS. Even worse, due to the increased complexity of the algorithm, its theoretical analysis is likely to differ from the practical results.\\

The problem at hand is to modify the current implementation of IQS and IIQS to support an extra case, which is when the sequence of elements can have repeated elements on it. This is now a problem for many reasons as it messes up the pivot selection heuristics and partition stages of the original algorithm.\\

Due to the aforementioned reasons, we want to take an experimental approach to analyze this new instance and use the results to guide the development of an extension of this algorithm.\\

\FloatBarrier
\section{Methodology foundationals}
\label{SECTION:METHODOLOGY_FOUNDATIONALS}

\subsection{Algorithm instantiation hierarchy}
The main goal of this work is to devise if we can design version of both IQS and IIQS which can avoid the worst case when dealing with sequences that hold repeated elements. Once this goal is accomplished, then we need to study its behaviour in order to check ways to deliver the same performance for both repeating and non repeating sequences. Thus our instantiation hierarchy is as follows:\\

\subsubsection{Metaheuristics and algorithm paradigms}
IncrementalQuickSelect and IntrospectiveIncrementalQuickSelect are algorithms used for partial sorting, belonging to the \textit{partition-based adaptive sorting} family which are our optimization target using repeated elements on a sequence.\\

\subsubsection{Algorithms}
As both IQS and IIQS are partition-based algorithms, they both share common elements and routines, namely:\\

\begin{itemize}
    \item \textbf{Next}:This is the main process on both IQS and IIQS. It performs a minima extraction. It expected average running time is $O(n + log_2{n})$ on which $n$ is the size of the sequence being passed as input.
    \item \textbf{Partition}: Partitions a given sequence into three subsequences $p_1$, $p_2$ and $p_3$ which follows that $\forall~p_i \in p_1,~\forall~p_j \in p_2:~ p_i < p_j$ and $\forall~p_i \in p_3,~\forall~p_j \in p_2:~p_i < p_j$. It expected average running time is $O(m)$ on which $m$ is the size of the sequence being passed as input following $m \leq n$ on which $n$ is the total sequence length.
    \item \textbf{Swap}: Swaps two elements in the sequence in-place. It expected average running time is $O(1)$.
    \item \textbf{PushStack}: Pushes an element into the stack. It expected average running time is $O(1)$.
    \item \textbf{PullStack}: Pulls an element from the stack. It expected average running time is $O(1)$.
\end{itemize}

As for IIQS exclusive use routines we can mention:
\begin{itemize}
    \item \textbf{BFPRT:} Implementation of median of medians algorithm~\cite{Blum_Floyd_Pratt_Rivest_Tarjan_1973}. This algorithm is used as a fallback option for the random selection pivot selection performed during each iteration of IQS. It expected average running time is $O(m)$ on which $m \leq n$ is the size of the sequence being passed as input.
    \item \textbf{Median:} Sorts in-place an array of fixed size\footnote{Whilst on literature a median of medians of five elements is suggested, we do not want to tie the implementation of the algorithm to a fixed value, but rather become this size a parameter of our algorithm.} and then retrieves the element in the middle position. It expected average running time is $O(1)$, despite the complexity of the sorting mechanism used as the time used is constant and is not in function of the sequence length.
\end{itemize}

\subsubsection{Source program}
As for the implementation, our language of choice was C++ in conjuntion with Boost libraries for argument parsing. \\

\subsubsection{Object code}
Object code is obtained via direct compilation of the main source file. Due to portability concerns, this process is triggered via Makefiles but no special or separate treatment is done for the compilation artifacts.\\

\subsubsection{Process}
All experiments are executed on userspace without any extra execution privileges.\\

\subsection{Algorithm design hierarchy}

\subsubsection{System structure}
Experiments are structured in a way that prioritizes execution flexibility over convention. Initial versions of the implemented algorithm were developed using C, but this language as is has some problems when it comes to understanding from the developer's standpoint. In this aspect, C++ (which can be seen as an class-oriented version of C) helps to perform an easier modularizaton of the code, while maintaining the macro flexibility of C. Apart from the main algorithm implementations, we can identify the following structures:\\

\subsubsection{Main driver}
The entrypoint of our compilation unit is our daily driver, which takes charge of argument parsing, main execution and high-level operations as snapshot dumping.  Runtime options are parsed with the help of Boost's \texttt{boost::program\_options} library, which takes charge of parsing and converting arguments to their corresponding types. \\

The list of available arguments to pass to the main driver are shown in the table shown in Table~\ref{TABLE:ARGUMENTS}:\\


\begin{table}[!ht]
    \centering

    \begin{tabularx}{\linewidth}{|r|r|X|}
        \hline
        STD type & CLI option & Description \\
        \hline
        \texttt{bool} & \texttt{--log\_pivot\_time} & Enables clock for pivot selection runtime\\
        \hline
        \texttt{bool} & \texttt{--log\_iteration\_time} & Enables clock for iteration runtime \\
        \hline
        \texttt{bool} & \texttt{--log\_extraction\_time} & Enables clock for element extraction runtime\\
        \hline
        \texttt{bool} & \texttt{--log\_swaps} & Enables logging of swap information \\
        \hline
        \texttt{bool} & \texttt{--use\_bfprt} & Enables median of medians instead of random selection (only for IIQS) \\
        \hline
        \texttt{bool} & \texttt{--use\_iiqs} & Enables use of IIQS instead of IQS \\
        \hline
        \texttt{bool} & \texttt{--use\_random\_pivot} & Enables random pivot seletion \\
        \hline
        \texttt{bool} & \texttt{--enable\_reuse} & Enables pivot reuse \\
        \hline
        \texttt{double} & \texttt{--alpha\_value} & Sets $\alpha$ value \\
        \hline
        \texttt{double} & \texttt{--beta\_value} & Sets $\beta$ value \\
        \hline
        \texttt{double} & \texttt{--pivot\_bias} & Sets pivot selection bias \\
        \hline
        \texttt{int} & \texttt{--random\_seed\_value} & Sets random seed value \\
        \hline
        \texttt{std::size\_t} & \texttt{--input\_size} & Experiment input size \\
        \hline
        \texttt{std::size\_t} & \texttt{--extractions} & Experiment extractions to perform \\
        \hline
        \texttt{std::string} & \texttt{--input\_file\_value} & Input file path \\
        \hline
        \texttt{std::string} & \texttt{--output\_file\_value} & Output file path \\
        \hline    
    \end{tabularx}
    \caption{Arguments for main driver program}
    \label{TABLE:ARGUMENTS}
\end{table}

\subsubsection{Snapshot spec}
The information about the program execution is stored on \textit{snapshots}, which are inspired on Valgrind profiler stats. Snapshot contains a rather large amount of information, from program configurations, options, and current state. In order to capture snapshots, a set of precompiler macros has been defined to enable timing of whole routines, partial sections, and conditional values. \\

In order to minimize the performance hit, there is only one instance of \texttt{snapshot\_t} being kept on memory at all times which is passed as reference to the instances of IQS and IIQS. This snapshot is only saved on the record array log on RAM at certain keypoints. This record array log is preinitialized before the execution in order to avoid allocation calls during the execution phase. Currently the time unit used for benchmarking is \texttt{std::chrono::nanoseconds}, defined on the \texttt{TIME\_UNIT} macro.\\

The table on Table~\ref{TABLE:SNAPSHOT_STRUCTURE} shows the logged metrics described on the \texttt{snapshot\_t} structure:\\

\begin{table}[!ht]
    \centering
    \begin{tabularx}{\linewidth}{|r|X|}%|X|}
    \hline
    STD type & logged variable \\ % & Description \\ %& \\
    \hline
    \texttt{TIME\_UNIT} & \texttt{iteration\_time} \\ %& \\
    \hline
    \texttt{TIME\_UNIT} & \texttt{total\_iteration\_time} \\ %& \\
    \hline
    \texttt{TIME\_UNIT} & \texttt{partition\_time} \\ %& \\
    \hline
    \texttt{TIME\_UNIT} & \texttt{total\_partition\_time} \\ %& \\
    \hline
    \texttt{TIME\_UNIT} & \texttt{bfprt\_partition\_time} \\ %& \\
    \hline
    \texttt{TIME\_UNIT} & \texttt{total\_bfprt\_partition\_time} \\ %& \\
    \hline
    \texttt{TIME\_UNIT} & \texttt{extraction\_time} \\ %& \\
    \hline
    \texttt{TIME\_UNIT} & \texttt{total\_extraction\_time} \\ %& \\
    \hline
    \texttt{size\_t} & \texttt{current\_extraction\_executed\_partitions} \\ %& \\
    \hline
    \texttt{size\_t} & \texttt{total\_executed\_partitions} \\ %& \\
    \hline
    \texttt{size\_t} & \texttt{current\_iteration\_partition\_swaps} \\ %& \\
    \hline
    \texttt{size\_t} & \texttt{total\_executed\_partition\_swaps} \\ %& \\
    \hline
    \texttt{double} & \texttt{current\_iteration\_longest\_partition\_swap} \\ %& \\
    \hline
    \texttt{double} & \texttt{total\_executed\_longest\_partition\_swap} \\ %& \\
    \hline
    \texttt{size\_t} & \texttt{current\_iteration\_executed\_bfprt\_partitions} \\ %& \\
    \hline
    \texttt{size\_t} & \texttt{total\_executed\_bfprt\_partitions} \\ %& \\
    \hline
    \texttt{size\_t} & \texttt{current\_iteration\_bfprt\_partition\_swaps} \\ %& \\
    \hline
    \texttt{size\_t} & \texttt{total\_executed\_bfprt\_partition\_swaps} \\ %& \\
    \hline
    \texttt{double} & \texttt{current\_iteration\_longest\_bfprt\_partition\_swap} \\ %& \\
    \hline
    \texttt{double} & \texttt{total\_executed\_longest\_bfprt\_partition\_swap} \\ %& \\
    \hline
    \texttt{double} & \texttt{current\_extracted\_pivot} \\ %& \\
    \hline
    \texttt{size\_t} & \texttt{current\_stack\_size} \\ %& \\
    \hline
    \texttt{size\_t} & \texttt{total\_pushed\_pivots} \\ %& \\
    \hline
    \texttt{size\_t} & \texttt{total\_pulled\_pivots} \\ %& \\
    \hline
    \texttt{size\_t} & \texttt{current\_iteration\_pushed\_pivots} \\ %& \\
    \hline
    \texttt{size\_t} & \texttt{current\_iteration\_pulled\_pivots} \\ %& \\
    \hline
    \texttt{size\_t} & \texttt{current\_extraction} \\ %& \\
    \hline
    \texttt{size\_t} & \texttt{input\_size} \\ %& \\
    \hline
    \texttt{char} & \texttt{snapshot\_point} \\ %& \\
    \hline
\end{tabularx}
\caption{Snapshot structure}
\label{TABLE:SNAPSHOT_STRUCTURE}
\end{table}


\subsubsection{Algorithm and data structure design}
Testbench implementation is divided into four major components:\\

\begin{itemize}
    \item{
        \textbf{IQS C++ Implementation}: Base C++ implementation of IQS with support for C++ STD container classes. One of the differences with the standard implementation of IQS is the presence of \texttt{IQS::random\_between} and \texttt{IQS::biased\_between} methods, which allows control over the pivot selection methods.
    }
    \item{
        \textbf{IIQS C++ Implementation}: Base C++ implementation of IIQS with support for C++ STD container classes. Inherits all components for IQS so this implementation only overloads \texttt{IQS::next} method and adds \texttt{IIQS::bfprt}, intended to support the extra operations needed by IIQS.
    }
    \item{
        \textbf{IQS low-level C++ Implementation}: C++ implementation of IQS without support for C++ STD container classes, relying only on direct memory allocation. This implementation was not benchmarked as it is only intended to be used as reference.
    }
    \item{
        \textbf{IIQS low-level C++ Implementation}: C++ implementation of IIQS without support for C++ STD container classes, relying only on direct memory allocation. This implementation was not benchmarked as it is only intended to be used as reference. All methods from low-level IQS are inherited here.
    }
\end{itemize}

\subsubsection{Implementation and algorithm tuning}
On the original IIQS analysis, randomized sequences and sorted sequences were used as tests. The original problem constrained all worst case input instances to be ordered sequencies in order to ease understanding. On ordered sequences is easier to see when a pivot selection fails by misusing the stack, thus not reducing the problem size. But since we now are dealing with repeated elements, new sequences for input are needed to test such cases.\\

\begin{itemize}
        
    \item{\textbf{Randomized sequences}: 
    This is our classical test case, on which all the elements are shuffled without any special criteria.}

    \item{\textbf{Ascending sequences}: 
    Used to generate a synthetic worst-case instance for IQS, this sequence is ordered in ascending order.}

    \item{\textbf{Descending sequences}: 
    Used to generate a synthetic worst-case instance for IQS, this sequence is ordered in descending order.}

    \item{\textbf{Constrained classes}: 
    Given $m < n$ the number of classes on the sequence, we want to test the effect of the ratio $\frac{m}{n}$ for a fixed number classes to devise if there is a relationship between the number of classes and the running time of the algorithm. This input is shuffled after its generation.}

    \item{\textbf{Constrained classes with random noise}: 
    In addition to the previous instance, we also add a random number of elements which do not belong to any instances of $m$ to induce random noise on the sample. This input is shuffled after its generation.}

    \item{\textbf{Shuffled sequences with sorted segments}: 
    Based on a mix of \textit{Runs}, \textit{SUS} and \textit{SMS.SUS} metrics for adaptive sorting, this input attempts to test the effect of presortedness on the execution of IQS and IIQS. To generate this input we first generate a shuffled input and then for each subsegment of the shuffled sequence we execute a partial sorting.}

    \item{\textbf{Randomized sequence with ignored noise}: 
    This input is intended to test if discontinuities on the sorting process can affect the performance by ignoring certain swaps. To accomplish this, we take a randomized sequence and then for a given amount of elements on the sequence we put the value that belong to their position.}
\end{itemize}

Aditionally, due to the nature of the problem, we have decided to constrain the following two aspects of the implementation in order to ensure performance and replication of results. So, they can be peer-validated at a later stage. Replication is achieved by taking a monte-carlo simulation~\cite{10.5555/1614191} approach using the following means:\\

\begin{itemize}
    \item{\textbf{Fixed seeding}
    For all experiments, all inputs are generated beforehand on the same instance of the machine in sequential order and providing the same seed for all random number generators.}
    
    \item{\textbf{Systematic randomization}
    For all processes that require randomization, the random values provided are extracted from a separate file beforehand, this ensures that all extractions of random numbers for the use of the algorithm are delivered in the same order across executions.}
\end{itemize}

\subsubsection{Code tuning}
%only one instance
%preinstanciation before clocking
\begin{itemize}
    \item{\textbf{Unique snapshot intance}: 
    In order to minmize memory consumption and allocation operations only one snapshot instance is initialized for a whole experiment, and it is passed as reference along the whole program.}
    \item{\textbf{Memory pre-allocation}: 
    All test cases, files, snapshot space, and random generated number are computed by an external process and they are fed into the program via file inputs which are read using STL \texttt{std::ifstream} and initialized before all tests start, so memory is allocated already at this point in order to prevent reallocation operations.}
    \item{\textbf{Unique source of truth}:
    All random numbers used along implementations are extracted from a unique source, from the same allocated space during runtime. All alocations are performed before the main execution and clocks start running in order to ensure that inicialization process does not affect runtimes due to memory allocation overhead.}
\end{itemize}

\subsubsection{System software}
Our compiler is GNU GCC 9.3.0 configured for a x86\_64-linux-gnu target with posix thread modeling. All compiler optimizations are disabled in order to track time more acurrately.
\subsubsection{Platform and hardware}
The specs of the machine used are shown in the table on Table~\ref{TABLE:SPECS}:\\

\begin{table}[!ht]
    \centering
    \begin{tabularx}{\linewidth}{|l|X|}
        \hline
        Item & Product ID \\
        \hline
        Processor  & Ryzen 5 Series 3600, 6C/12T 3.6 GHz base processor clock 4.2 GHz max boost, 35MB cache, unlocked clock settings \\
        \hline
        Memory  & Team T-FORCE DARK Za 32GB (2 x 16GB) DDR4 3600 MHz (PC4 28800) TDZAD432G3600HC18JDC01, dual-channel enabled, XMP profile 1 enabled\\
        \hline
        Storage  & WD M.2 SSD 480GB WDS480G2G0B \\
        \hline
        Motherboard  & MSI B450M PRO-VDH MAX, AM4 \\
        \hline
        Power Suppy Units  & EVGA 600W W1, 80+ Certified\\
        \hline
        Video Adapter  &  Galax Video NVIDIA GeForce GTX1650 1-Click OC  \\
        \hline
        Operating System  &  Pop!\_OS 20.04 LTS  \\
        \hline
        Kernel  &  Linux rspk-shoukugun 5.4.0-7629-generic \#33\~{}1589834512\~{}20.04\~{}ff6e79e-Ubuntu SMP Mon May 18 23:29:32 UTC  x86\_64 x86\_64 x86\_64 GNU/Linux\\
        \hline
        Linux Version  & Linux version 5.4.0-7629-generic (buildd@lcy01-amd64-013) (gcc version 9.3.0 (Ubuntu 9.3.0-10ubuntu2)) \#33\~{}1589834512\~{}20.04\~{}ff6e79e-Ubuntu SMP Mon May 18 23:29:32 UTC \\ 
        \hline
    \end{tabularx}    
    \caption{Test machine specs}
    \label{TABLE:SPECS}
\end{table}
\FloatBarrier
\section{Experimental process breakdown}
% first generate inputs
% perform gridsearch
% analyze results
% present
% raise next question
Now, we present a breakdown of the experimental process definitions needed in order to begin executing our experiments.\\

\subsection{Experimental cycle}
As mentioned before in Section~\ref{SUBSECTION:EXPERIMENTAL_ALGORITMICS_CONSIDERATIONS}, this process is being handled in a cyclic manner, and each experiment is controlled by their own Jupyter Notebook. Now we describe the steps followed through the realization of the experiments of this report.\\

\subsubsection{Hypothesis}
We begin by raising a question and a doable answer to it which we want to prove if it holds for the current experimentation cycle. In the context of this report, hypothesis are code tuning improvements that we want to check before building a solution for our problem.\\

\subsubsection{Input and execution setup}
After defining our topic, we start generating inputs and planning the experiment execution. As for the inputs, we already defined in Section~\ref{SECTION:METHODOLOGY_FOUNDATIONALS} how the inputs are be generated, as for each experiment inputs are to be selected accordingly on what we want to test or explore. \\

On the other hand, as the main driver already accepts a defined set of parameters to control the execution, those are used to setup the experiment environment. From now on, we refer to the cross product of the combination of inputs and program parameters as the \textit{search space}.\\

\subsubsection{Execution}
At this point we execute a GridSearch over out \textit{search space} in sequential order.This way we reduce the amount of disturbances that the experiment can suffer. After the execution of the experiments, snapshots are stored on a single ASCII file using a comma separated schema, which is is later used for the analysis.\\

\subsubsection{Analysis}
The results are gathered on Jupyter and examinated in order to get insight on the phenomena and to check if the hypothesis is valid or not. After we gather enough information, a discussion on the results is held, which is written together with the results on this report.\\

\subsection{Metrics and indicators}
In order to evaluate the experiments, we need to define beforehand some metrics to determine which aspects of the execution are evaluated during the experiments. The difference between comparing raw data and use metrics is the pre-processing being made in order to gain useful insight about what is going on under the hood before executing our first benchmark.\\

\subsubsection{Swaps}
Using the definitions for measuring disorder given in Section \ref{SEC:MEASURING_DISORDER}, we can establish metric for complete sorting algorithms. Both IQS and IIQS falls into the definition of incremental sorting as shown on Section~\ref{SEC:INCREMENTAL_SORTING}, making its use natural to us. But as our study consists on analysing the behaviour of extractions and not of a complete sorting execution, such metrics are unrealistic to such purposes and unnatural.\\

Still, such definitions can help us to establish a base to define our own disorder metric given the following known premises:\\

\begin{itemize}
    \item Both IQS and IIQS rely on \textit{partition} in order to perform the partial sorting the same way as \textit{QuickSort}.
    \item QuickSort, being a adaptive sorting algorithm, is influenced by presortedness.
    \item \textit{Dis}, \textit{Max}, \textit{Exc}, \textit{Rem}, \textit{Runs}, \textit{SUS}, and \textit{SMS.SUS} are applicable metrics for \textit{QuickSort}.
    \item Both IQS and IIQS perform their heavy lifting at the partition stage.
\end{itemize}

Then, the minimum common denominator of the aforementioned premises is the behaviour of swap operations. Given the nature of \textit{partition} operation, it is expected to not exchange any elements (\textit{Exc}) on a sorted sequence, and transitively, each segment of every iteration of IQS must follow the same property given that it is being performed in-place.\\

On the other hand, as it can be seen on the executions for worst case of IQS, as there are long ascending sub-sequences on the input (\textit{Runs}, \textit{SUS}, \textit{SMS.SUS}), the fastest the partition stage ends, as it is not performing any swaps and in some cases not storing any pivots at all.\\

Now when a swap operation occurs at the partition stage, the elements being swapped and the pivot share a partial sorting relationship between them. Given that this relationship can occur at any point of the sequence we can state the following aspects of the swap operation:\\

\begin{itemize}
    \item The longest the distance of the swap being performed from the pivot, the elements are more far away from their actual position (\textit{Dis}, \textit{Max}).
    \item A sorted sequence does not perform any swaps on their partition stage.
    \item Given that all elements are being sorted in-place, and partitions are executed in a recursive way respect the pivot position, the previous two properties are transitive respect the partitions that are being generated during the extraction of a minima.
\end{itemize}

Then, we establish the following two disorder metrics for IQS and IIQS:\\

\begin{itemize}
    \item \textbf{N\_SWAPS}: As the number of swaps performed for a given iteration of IQS. This metric can be extended as cumulative regarding the extraction of a minima. The values for \textbf{N\_SWAPS} are in the range $[0, n^2]$ for a given iteration.
    \item \textbf{MAX\_SWAP}: As the longest swap performed during a iteration of IQS. This metric only applies to each individual execution of partition, as the maximum distance is bound to decrease on each iteration. The values for \textbf{MAX\_SWAP} are in the range $[0, m]$, on which $m$ is the size of the current partition which follows $m \leq n$.
\end{itemize}

\subsubsection{Stack operations}
A key aspect of IQS and IIQS is the amount of extra memory needed to perform the sorting. Under normal conditions, IQS requires $log_2(n)$ extra space in order to maintain all pivots for the first extraction. This makes the first extraction the most time-consuming operation as it partitions over $n$ elements and pushes $log_2(n)$ pivots into the stack in average.\\

One of the problems of IQS was the fact that in certain cases, $n$ pivots get stored on the stack, forcing for the non-introspective version to fix the stack size at $n$, whilst the introspective version due to its most stable behaviour can be safely set at $log_{1.7}(n)$ thanks to the median-of-median algorithm effect.\\

In light of the aforementioned facts, it makes sense that any modification being made to IQS or IIQS must also compare the performance of the stack growth due to caching and memory consumption concerns. In contrast to the previous work on IIQS, this version of the analysis also considers other extra metrics such as number of pulled elements and number of pushed elements per iteration and per minima extraction, as their behaviour is directly connected to the time used by the partitioning stage.\\

\subsubsection{Number of executed subroutines}
In line with the stack problem, it is sane to establish if the number of elements in the stack is in direct relation to the executions of the partition routine, and in the same way, the executions of the partition routine has direct relationship with the number of elements extracted. In this regard, we also want to log the time that both routines get called during minima extraction to study if they had any effect on the total running time of IQS and IIQS.\\

\subsubsection{Pivot bias}
The preferred way of testing worst-case executions on partition-based sorting algorithms is to fix the pivot selection to a position which ensures the worst outcome each time. This is the main reason on why introduction of randomization for selection for pivots is so effective on maintaining expected average case on such algorithms. For IQS, worst case comes from choosing the lowest or the highest element on the sequence which can be accomplished by fixing the pivot position on the edges of the sequence to be partitioned, given that the entire sequence is already sorted.\\

But when testing synthetic , it is not proven if there is a position which performs better for certain cases than selecting the middle element as pivot on partition based algorithms (given the same constraints as the previous paragraph), nor if this bias for pivot selection can be used to tune the algorithm beforehand for certain cases.\\

\subsubsection{Clocked routines}
Not all parts of the program are subjected to monitoring via snapshots, as this approach is both nonsensical and non practical. We just change the execution of the program on certain points of the execution in order to gather metrics or to push elements to the log array. \\

Currently defined sections to be used as snapshot points are shown in the table on Table~\ref{TABLE:POINTS}:

\begin{table}[!ht]
    \centering
    \begin{tabularx}{\linewidth}{|r|r|X|}
        \hline
        Program flag & code & Description \\
        \hline
        \texttt{EXTRACTION\_STAGE\_BEGIN} & \texttt{10} & Start of the minima extraction\\
        \hline
        \texttt{EXTRACTION\_STAGE\_END} & \texttt{20} & End of the minima extraction \\
        \hline
        \texttt{ITERATION\_STAGE\_BEGIN} & \texttt{30} & Begin of a IQS or IIQS iteration \\
        \hline
        \texttt{ITERATION\_STAGE\_LOOP} & \texttt{40} & Middle point of a IQS or IIQS iteration \\
        \hline
        \texttt{ITERATION\_STAGE\_INTROSPECT} & \texttt{41} & Introspect stage of IIQS \\
        \hline
        \texttt{ITERATION\_STAGE\_END} & \texttt{50} & End of IQS or IIQS iteration \\
        \hline
        \texttt{PARTITION\_STAGE\_BEGIN} & \texttt{60} & Start of partitioning stage \\
        \hline
        \texttt{PARTITION\_STAGE\_END} & \texttt{70} & End of partitioning stage \\
        \hline
    \end{tabularx}
    
    \caption{Timed sections}
    \label{TABLE:POINTS}
\end{table}

\FloatBarrier


% \section{Hiearchies of algorithm design}
% \section{Experimental process}

\chapter{Methodology}
\chapterimage[height=3cm]{./fragments/memes/thor_tools}
\label{CHAPTER:METHODLOGY}

In this chapter we revisit the methodology explained in Section~\ref{SECTION:EXPERIMENTAL_ALGORITHMICS} in order to understand better how it is applied to our problem at hand.


\section{Rationale}
IncrementalQuickSort and its introspective version, IntrospectiveQuickSort, already have their theoretical analyses for the worst case instances. But such theoretical analysis is not always feasible, sometimes not easy and most of the time it is not realistic. As a practical example of it, when testing IQS against HeapSort for a full array sorting under architectures with small cache memory, IQS outperforms vastly HeapSort as it trashes the cache on each iteration. But when cache units are large enough to support the entire array, there is no point on using IQS, as most operations are actually solved on cache directly. There is a huge gap when it comes to practice on algorithm design, and IIQS is also not free of such problems.\\

The main issue arose when it comes to the analysis of the median-of-medians effects on the partition. As the execution of this algorithm offsets itself on each partition, it is the equivalent to run continuously a process which reduces the overall disorder of the sequence\footnote{We do not talk explicitly of any disorder metric seen in Section~\ref{SEC:MEASURING_DISORDER} as the effect depends on the process behind each execution. In this sense, we want our definition to be abstract and not to be tied with any algorithm implementation.}.\\

This effect displaces the elements on the sequence towards their expected position on it, due to the adaptive-sorting nature of both IQS and IIQS makes the overall running time dependent on the element distribution. This makes really hard the use of standard techniques like amortized analysis to study the behavior of IIQS. Even worse, due to the increased complexity of the algorithm, its theoretical analysis is likely to differ from the practical results.\\

The problem at hand is to modify the current implementation of IQS and IIQS to support an extra case, which is when the sequence of elements can have repeated elements on it. This is now a problem for many reasons as it messes up the pivot selection heuristics and partition stages of the original algorithm.\\

Due to the aforementioned reasons, we want to take an experimental approach to analyze this new instance and use the results to guide the development of an extension of this algorithm.\\

\FloatBarrier
\section{Methodology foundationals}
\label{SECTION:METHODOLOGY_FOUNDATIONALS}

\subsection{Algorithm instantiation hierarchy}
The main goal of this work is to devise if we can design version of both IQS and IIQS which can avoid the worst case when dealing with sequences that hold repeated elements. Once this goal is accomplished, then we need to study its behaviour in order to check ways to deliver the same performance for both repeating and non repeating sequences. Thus our instantiation hierarchy is as follows:\\

\subsubsection{Metaheuristics and algorithm paradigms}
IncrementalQuickSelect and IntrospectiveIncrementalQuickSelect are algorithms used for partial sorting, belonging to the \textit{partition-based adaptive sorting} family which are our optimization target using repeated elements on a sequence.\\

\subsubsection{Algorithms}
As both IQS and IIQS are partition-based algorithms, they both share common elements and routines, namely:\\

\begin{itemize}
    \item \textbf{Next}:This is the main process on both IQS and IIQS. It performs a minima extraction. It expected average running time is $O(n + log_2{n})$ on which $n$ is the size of the sequence being passed as input.
    \item \textbf{Partition}: Partitions a given sequence into three subsequences $p_1$, $p_2$ and $p_3$ which follows that $\forall~p_i \in p_1,~\forall~p_j \in p_2:~ p_i < p_j$ and $\forall~p_i \in p_3,~\forall~p_j \in p_2:~p_i < p_j$. It expected average running time is $O(m)$ on which $m$ is the size of the sequence being passed as input following $m \leq n$ on which $n$ is the total sequence length.
    \item \textbf{Swap}: Swaps two elements in the sequence in-place. It expected average running time is $O(1)$.
    \item \textbf{PushStack}: Pushes an element into the stack. It expected average running time is $O(1)$.
    \item \textbf{PullStack}: Pulls an element from the stack. It expected average running time is $O(1)$.
\end{itemize}

As for IIQS exclusive use routines we can mention:
\begin{itemize}
    \item \textbf{BFPRT:} Implementation of median of medians algorithm~\cite{Blum_Floyd_Pratt_Rivest_Tarjan_1973}. This algorithm is used as a fallback option for the random selection pivot selection performed during each iteration of IQS. It expected average running time is $O(m)$ on which $m \leq n$ is the size of the sequence being passed as input.
    \item \textbf{Median:} Sorts in-place an array of fixed size\footnote{Whilst on literature a median of medians of five elements is suggested, we do not want to tie the implementation of the algorithm to a fixed value, but rather become this size a parameter of our algorithm.} and then retrieves the element in the middle position. It expected average running time is $O(1)$, despite the complexity of the sorting mechanism used as the time used is constant and is not in function of the sequence length.
\end{itemize}

\subsubsection{Source program}
As for the implementation, our language of choice was C++ in conjuntion with Boost libraries for argument parsing. \\

\subsubsection{Object code}
Object code is obtained via direct compilation of the main source file. Due to portability concerns, this process is triggered via Makefiles but no special or separate treatment is done for the compilation artifacts.\\

\subsubsection{Process}
All experiments are executed on userspace without any extra execution privileges.\\

\subsection{Algorithm design hierarchy}

\subsubsection{System structure}
Experiments are structured in a way that prioritizes execution flexibility over convention. Initial versions of the implemented algorithm were developed using C, but this language as is has some problems when it comes to understanding from the developer's standpoint. In this aspect, C++ (which can be seen as an class-oriented version of C) helps to perform an easier modularizaton of the code, while maintaining the macro flexibility of C. Apart from the main algorithm implementations, we can identify the following structures:\\

\subsubsection{Main driver}
The entrypoint of our compilation unit is our daily driver, which takes charge of argument parsing, main execution and high-level operations as snapshot dumping.  Runtime options are parsed with the help of Boost's \texttt{boost::program\_options} library, which takes charge of parsing and converting arguments to their corresponding types. \\

The list of available arguments to pass to the main driver are shown in the table shown in Table~\ref{TABLE:ARGUMENTS}:\\


\begin{table}[!ht]
    \centering

    \begin{tabularx}{\linewidth}{|r|r|X|}
        \hline
        STD type & CLI option & Description \\
        \hline
        \texttt{bool} & \texttt{--log\_pivot\_time} & Enables clock for pivot selection runtime\\
        \hline
        \texttt{bool} & \texttt{--log\_iteration\_time} & Enables clock for iteration runtime \\
        \hline
        \texttt{bool} & \texttt{--log\_extraction\_time} & Enables clock for element extraction runtime\\
        \hline
        \texttt{bool} & \texttt{--log\_swaps} & Enables logging of swap information \\
        \hline
        \texttt{bool} & \texttt{--use\_bfprt} & Enables median of medians instead of random selection (only for IIQS) \\
        \hline
        \texttt{bool} & \texttt{--use\_iiqs} & Enables use of IIQS instead of IQS \\
        \hline
        \texttt{bool} & \texttt{--use\_random\_pivot} & Enables random pivot seletion \\
        \hline
        \texttt{bool} & \texttt{--enable\_reuse} & Enables pivot reuse \\
        \hline
        \texttt{double} & \texttt{--alpha\_value} & Sets $\alpha$ value \\
        \hline
        \texttt{double} & \texttt{--beta\_value} & Sets $\beta$ value \\
        \hline
        \texttt{double} & \texttt{--pivot\_bias} & Sets pivot selection bias \\
        \hline
        \texttt{int} & \texttt{--random\_seed\_value} & Sets random seed value \\
        \hline
        \texttt{std::size\_t} & \texttt{--input\_size} & Experiment input size \\
        \hline
        \texttt{std::size\_t} & \texttt{--extractions} & Experiment extractions to perform \\
        \hline
        \texttt{std::string} & \texttt{--input\_file\_value} & Input file path \\
        \hline
        \texttt{std::string} & \texttt{--output\_file\_value} & Output file path \\
        \hline    
    \end{tabularx}
    \caption{Arguments for main driver program}
    \label{TABLE:ARGUMENTS}
\end{table}

\subsubsection{Snapshot spec}
The information about the program execution is stored on \textit{snapshots}, which are inspired on Valgrind profiler stats. Snapshot contains a rather large amount of information, from program configurations, options, and current state. In order to capture snapshots, a set of precompiler macros has been defined to enable timing of whole routines, partial sections, and conditional values. \\

In order to minimize the performance hit, there is only one instance of \texttt{snapshot\_t} being kept on memory at all times which is passed as reference to the instances of IQS and IIQS. This snapshot is only saved on the record array log on RAM at certain keypoints. This record array log is preinitialized before the execution in order to avoid allocation calls during the execution phase. Currently the time unit used for benchmarking is \texttt{std::chrono::nanoseconds}, defined on the \texttt{TIME\_UNIT} macro.\\

The table on Table~\ref{TABLE:SNAPSHOT_STRUCTURE} shows the logged metrics described on the \texttt{snapshot\_t} structure:\\

\begin{table}[!ht]
    \centering
    \begin{tabularx}{\linewidth}{|r|X|}%|X|}
    \hline
    STD type & logged variable \\ % & Description \\ %& \\
    \hline
    \texttt{TIME\_UNIT} & \texttt{iteration\_time} \\ %& \\
    \hline
    \texttt{TIME\_UNIT} & \texttt{total\_iteration\_time} \\ %& \\
    \hline
    \texttt{TIME\_UNIT} & \texttt{partition\_time} \\ %& \\
    \hline
    \texttt{TIME\_UNIT} & \texttt{total\_partition\_time} \\ %& \\
    \hline
    \texttt{TIME\_UNIT} & \texttt{bfprt\_partition\_time} \\ %& \\
    \hline
    \texttt{TIME\_UNIT} & \texttt{total\_bfprt\_partition\_time} \\ %& \\
    \hline
    \texttt{TIME\_UNIT} & \texttt{extraction\_time} \\ %& \\
    \hline
    \texttt{TIME\_UNIT} & \texttt{total\_extraction\_time} \\ %& \\
    \hline
    \texttt{size\_t} & \texttt{current\_extraction\_executed\_partitions} \\ %& \\
    \hline
    \texttt{size\_t} & \texttt{total\_executed\_partitions} \\ %& \\
    \hline
    \texttt{size\_t} & \texttt{current\_iteration\_partition\_swaps} \\ %& \\
    \hline
    \texttt{size\_t} & \texttt{total\_executed\_partition\_swaps} \\ %& \\
    \hline
    \texttt{double} & \texttt{current\_iteration\_longest\_partition\_swap} \\ %& \\
    \hline
    \texttt{double} & \texttt{total\_executed\_longest\_partition\_swap} \\ %& \\
    \hline
    \texttt{size\_t} & \texttt{current\_iteration\_executed\_bfprt\_partitions} \\ %& \\
    \hline
    \texttt{size\_t} & \texttt{total\_executed\_bfprt\_partitions} \\ %& \\
    \hline
    \texttt{size\_t} & \texttt{current\_iteration\_bfprt\_partition\_swaps} \\ %& \\
    \hline
    \texttt{size\_t} & \texttt{total\_executed\_bfprt\_partition\_swaps} \\ %& \\
    \hline
    \texttt{double} & \texttt{current\_iteration\_longest\_bfprt\_partition\_swap} \\ %& \\
    \hline
    \texttt{double} & \texttt{total\_executed\_longest\_bfprt\_partition\_swap} \\ %& \\
    \hline
    \texttt{double} & \texttt{current\_extracted\_pivot} \\ %& \\
    \hline
    \texttt{size\_t} & \texttt{current\_stack\_size} \\ %& \\
    \hline
    \texttt{size\_t} & \texttt{total\_pushed\_pivots} \\ %& \\
    \hline
    \texttt{size\_t} & \texttt{total\_pulled\_pivots} \\ %& \\
    \hline
    \texttt{size\_t} & \texttt{current\_iteration\_pushed\_pivots} \\ %& \\
    \hline
    \texttt{size\_t} & \texttt{current\_iteration\_pulled\_pivots} \\ %& \\
    \hline
    \texttt{size\_t} & \texttt{current\_extraction} \\ %& \\
    \hline
    \texttt{size\_t} & \texttt{input\_size} \\ %& \\
    \hline
    \texttt{char} & \texttt{snapshot\_point} \\ %& \\
    \hline
\end{tabularx}
\caption{Snapshot structure}
\label{TABLE:SNAPSHOT_STRUCTURE}
\end{table}


\subsubsection{Algorithm and data structure design}
Testbench implementation is divided into four major components:\\

\begin{itemize}
    \item{
        \textbf{IQS C++ Implementation}: Base C++ implementation of IQS with support for C++ STD container classes. One of the differences with the standard implementation of IQS is the presence of \texttt{IQS::random\_between} and \texttt{IQS::biased\_between} methods, which allows control over the pivot selection methods.
    }
    \item{
        \textbf{IIQS C++ Implementation}: Base C++ implementation of IIQS with support for C++ STD container classes. Inherits all components for IQS so this implementation only overloads \texttt{IQS::next} method and adds \texttt{IIQS::bfprt}, intended to support the extra operations needed by IIQS.
    }
    \item{
        \textbf{IQS low-level C++ Implementation}: C++ implementation of IQS without support for C++ STD container classes, relying only on direct memory allocation. This implementation was not benchmarked as it is only intended to be used as reference.
    }
    \item{
        \textbf{IIQS low-level C++ Implementation}: C++ implementation of IIQS without support for C++ STD container classes, relying only on direct memory allocation. This implementation was not benchmarked as it is only intended to be used as reference. All methods from low-level IQS are inherited here.
    }
\end{itemize}

\subsubsection{Implementation and algorithm tuning}
On the original IIQS analysis, randomized sequences and sorted sequences were used as tests. The original problem constrained all worst case input instances to be ordered sequencies in order to ease understanding. On ordered sequences is easier to see when a pivot selection fails by misusing the stack, thus not reducing the problem size. But since we now are dealing with repeated elements, new sequences for input are needed to test such cases.\\

\begin{itemize}
        
    \item{\textbf{Randomized sequences}: 
    This is our classical test case, on which all the elements are shuffled without any special criteria.}

    \item{\textbf{Ascending sequences}: 
    Used to generate a synthetic worst-case instance for IQS, this sequence is ordered in ascending order.}

    \item{\textbf{Descending sequences}: 
    Used to generate a synthetic worst-case instance for IQS, this sequence is ordered in descending order.}

    \item{\textbf{Constrained classes}: 
    Given $m < n$ the number of classes on the sequence, we want to test the effect of the ratio $\frac{m}{n}$ for a fixed number classes to devise if there is a relationship between the number of classes and the running time of the algorithm. This input is shuffled after its generation.}

    \item{\textbf{Constrained classes with random noise}: 
    In addition to the previous instance, we also add a random number of elements which do not belong to any instances of $m$ to induce random noise on the sample. This input is shuffled after its generation.}

    \item{\textbf{Shuffled sequences with sorted segments}: 
    Based on a mix of \textit{Runs}, \textit{SUS} and \textit{SMS.SUS} metrics for adaptive sorting, this input attempts to test the effect of presortedness on the execution of IQS and IIQS. To generate this input we first generate a shuffled input and then for each subsegment of the shuffled sequence we execute a partial sorting.}

    \item{\textbf{Randomized sequence with ignored noise}: 
    This input is intended to test if discontinuities on the sorting process can affect the performance by ignoring certain swaps. To accomplish this, we take a randomized sequence and then for a given amount of elements on the sequence we put the value that belong to their position.}
\end{itemize}

Aditionally, due to the nature of the problem, we have decided to constrain the following two aspects of the implementation in order to ensure performance and replication of results. So, they can be peer-validated at a later stage. Replication is achieved by taking a monte-carlo simulation~\cite{10.5555/1614191} approach using the following means:\\

\begin{itemize}
    \item{\textbf{Fixed seeding}
    For all experiments, all inputs are generated beforehand on the same instance of the machine in sequential order and providing the same seed for all random number generators.}
    
    \item{\textbf{Systematic randomization}
    For all processes that require randomization, the random values provided are extracted from a separate file beforehand, this ensures that all extractions of random numbers for the use of the algorithm are delivered in the same order across executions.}
\end{itemize}

\subsubsection{Code tuning}
%only one instance
%preinstanciation before clocking
\begin{itemize}
    \item{\textbf{Unique snapshot intance}: 
    In order to minmize memory consumption and allocation operations only one snapshot instance is initialized for a whole experiment, and it is passed as reference along the whole program.}
    \item{\textbf{Memory pre-allocation}: 
    All test cases, files, snapshot space, and random generated number are computed by an external process and they are fed into the program via file inputs which are read using STL \texttt{std::ifstream} and initialized before all tests start, so memory is allocated already at this point in order to prevent reallocation operations.}
    \item{\textbf{Unique source of truth}:
    All random numbers used along implementations are extracted from a unique source, from the same allocated space during runtime. All alocations are performed before the main execution and clocks start running in order to ensure that inicialization process does not affect runtimes due to memory allocation overhead.}
\end{itemize}

\subsubsection{System software}
Our compiler is GNU GCC 9.3.0 configured for a x86\_64-linux-gnu target with posix thread modeling. All compiler optimizations are disabled in order to track time more acurrately.
\subsubsection{Platform and hardware}
The specs of the machine used are shown in the table on Table~\ref{TABLE:SPECS}:\\

\begin{table}[!ht]
    \centering
    \begin{tabularx}{\linewidth}{|l|X|}
        \hline
        Item & Product ID \\
        \hline
        Processor  & Ryzen 5 Series 3600, 6C/12T 3.6 GHz base processor clock 4.2 GHz max boost, 35MB cache, unlocked clock settings \\
        \hline
        Memory  & Team T-FORCE DARK Za 32GB (2 x 16GB) DDR4 3600 MHz (PC4 28800) TDZAD432G3600HC18JDC01, dual-channel enabled, XMP profile 1 enabled\\
        \hline
        Storage  & WD M.2 SSD 480GB WDS480G2G0B \\
        \hline
        Motherboard  & MSI B450M PRO-VDH MAX, AM4 \\
        \hline
        Power Suppy Units  & EVGA 600W W1, 80+ Certified\\
        \hline
        Video Adapter  &  Galax Video NVIDIA GeForce GTX1650 1-Click OC  \\
        \hline
        Operating System  &  Pop!\_OS 20.04 LTS  \\
        \hline
        Kernel  &  Linux rspk-shoukugun 5.4.0-7629-generic \#33\~{}1589834512\~{}20.04\~{}ff6e79e-Ubuntu SMP Mon May 18 23:29:32 UTC  x86\_64 x86\_64 x86\_64 GNU/Linux\\
        \hline
        Linux Version  & Linux version 5.4.0-7629-generic (buildd@lcy01-amd64-013) (gcc version 9.3.0 (Ubuntu 9.3.0-10ubuntu2)) \#33\~{}1589834512\~{}20.04\~{}ff6e79e-Ubuntu SMP Mon May 18 23:29:32 UTC \\ 
        \hline
    \end{tabularx}    
    \caption{Test machine specs}
    \label{TABLE:SPECS}
\end{table}
\FloatBarrier
\section{Experimental process breakdown}
% first generate inputs
% perform gridsearch
% analyze results
% present
% raise next question
Now, we present a breakdown of the experimental process definitions needed in order to begin executing our experiments.\\

\subsection{Experimental cycle}
As mentioned before in Section~\ref{SUBSECTION:EXPERIMENTAL_ALGORITMICS_CONSIDERATIONS}, this process is being handled in a cyclic manner, and each experiment is controlled by their own Jupyter Notebook. Now we describe the steps followed through the realization of the experiments of this report.\\

\subsubsection{Hypothesis}
We begin by raising a question and a doable answer to it which we want to prove if it holds for the current experimentation cycle. In the context of this report, hypothesis are code tuning improvements that we want to check before building a solution for our problem.\\

\subsubsection{Input and execution setup}
After defining our topic, we start generating inputs and planning the experiment execution. As for the inputs, we already defined in Section~\ref{SECTION:METHODOLOGY_FOUNDATIONALS} how the inputs are be generated, as for each experiment inputs are to be selected accordingly on what we want to test or explore. \\

On the other hand, as the main driver already accepts a defined set of parameters to control the execution, those are used to setup the experiment environment. From now on, we refer to the cross product of the combination of inputs and program parameters as the \textit{search space}.\\

\subsubsection{Execution}
At this point we execute a GridSearch over out \textit{search space} in sequential order.This way we reduce the amount of disturbances that the experiment can suffer. After the execution of the experiments, snapshots are stored on a single ASCII file using a comma separated schema, which is is later used for the analysis.\\

\subsubsection{Analysis}
The results are gathered on Jupyter and examinated in order to get insight on the phenomena and to check if the hypothesis is valid or not. After we gather enough information, a discussion on the results is held, which is written together with the results on this report.\\

\subsection{Metrics and indicators}
In order to evaluate the experiments, we need to define beforehand some metrics to determine which aspects of the execution are evaluated during the experiments. The difference between comparing raw data and use metrics is the pre-processing being made in order to gain useful insight about what is going on under the hood before executing our first benchmark.\\

\subsubsection{Swaps}
Using the definitions for measuring disorder given in Section \ref{SEC:MEASURING_DISORDER}, we can establish metric for complete sorting algorithms. Both IQS and IIQS falls into the definition of incremental sorting as shown on Section~\ref{SEC:INCREMENTAL_SORTING}, making its use natural to us. But as our study consists on analysing the behaviour of extractions and not of a complete sorting execution, such metrics are unrealistic to such purposes and unnatural.\\

Still, such definitions can help us to establish a base to define our own disorder metric given the following known premises:\\

\begin{itemize}
    \item Both IQS and IIQS rely on \textit{partition} in order to perform the partial sorting the same way as \textit{QuickSort}.
    \item QuickSort, being a adaptive sorting algorithm, is influenced by presortedness.
    \item \textit{Dis}, \textit{Max}, \textit{Exc}, \textit{Rem}, \textit{Runs}, \textit{SUS}, and \textit{SMS.SUS} are applicable metrics for \textit{QuickSort}.
    \item Both IQS and IIQS perform their heavy lifting at the partition stage.
\end{itemize}

Then, the minimum common denominator of the aforementioned premises is the behaviour of swap operations. Given the nature of \textit{partition} operation, it is expected to not exchange any elements (\textit{Exc}) on a sorted sequence, and transitively, each segment of every iteration of IQS must follow the same property given that it is being performed in-place.\\

On the other hand, as it can be seen on the executions for worst case of IQS, as there are long ascending sub-sequences on the input (\textit{Runs}, \textit{SUS}, \textit{SMS.SUS}), the fastest the partition stage ends, as it is not performing any swaps and in some cases not storing any pivots at all.\\

Now when a swap operation occurs at the partition stage, the elements being swapped and the pivot share a partial sorting relationship between them. Given that this relationship can occur at any point of the sequence we can state the following aspects of the swap operation:\\

\begin{itemize}
    \item The longest the distance of the swap being performed from the pivot, the elements are more far away from their actual position (\textit{Dis}, \textit{Max}).
    \item A sorted sequence does not perform any swaps on their partition stage.
    \item Given that all elements are being sorted in-place, and partitions are executed in a recursive way respect the pivot position, the previous two properties are transitive respect the partitions that are being generated during the extraction of a minima.
\end{itemize}

Then, we establish the following two disorder metrics for IQS and IIQS:\\

\begin{itemize}
    \item \textbf{N\_SWAPS}: As the number of swaps performed for a given iteration of IQS. This metric can be extended as cumulative regarding the extraction of a minima. The values for \textbf{N\_SWAPS} are in the range $[0, n^2]$ for a given iteration.
    \item \textbf{MAX\_SWAP}: As the longest swap performed during a iteration of IQS. This metric only applies to each individual execution of partition, as the maximum distance is bound to decrease on each iteration. The values for \textbf{MAX\_SWAP} are in the range $[0, m]$, on which $m$ is the size of the current partition which follows $m \leq n$.
\end{itemize}

\subsubsection{Stack operations}
A key aspect of IQS and IIQS is the amount of extra memory needed to perform the sorting. Under normal conditions, IQS requires $log_2(n)$ extra space in order to maintain all pivots for the first extraction. This makes the first extraction the most time-consuming operation as it partitions over $n$ elements and pushes $log_2(n)$ pivots into the stack in average.\\

One of the problems of IQS was the fact that in certain cases, $n$ pivots get stored on the stack, forcing for the non-introspective version to fix the stack size at $n$, whilst the introspective version due to its most stable behaviour can be safely set at $log_{1.7}(n)$ thanks to the median-of-median algorithm effect.\\

In light of the aforementioned facts, it makes sense that any modification being made to IQS or IIQS must also compare the performance of the stack growth due to caching and memory consumption concerns. In contrast to the previous work on IIQS, this version of the analysis also considers other extra metrics such as number of pulled elements and number of pushed elements per iteration and per minima extraction, as their behaviour is directly connected to the time used by the partitioning stage.\\

\subsubsection{Number of executed subroutines}
In line with the stack problem, it is sane to establish if the number of elements in the stack is in direct relation to the executions of the partition routine, and in the same way, the executions of the partition routine has direct relationship with the number of elements extracted. In this regard, we also want to log the time that both routines get called during minima extraction to study if they had any effect on the total running time of IQS and IIQS.\\

\subsubsection{Pivot bias}
The preferred way of testing worst-case executions on partition-based sorting algorithms is to fix the pivot selection to a position which ensures the worst outcome each time. This is the main reason on why introduction of randomization for selection for pivots is so effective on maintaining expected average case on such algorithms. For IQS, worst case comes from choosing the lowest or the highest element on the sequence which can be accomplished by fixing the pivot position on the edges of the sequence to be partitioned, given that the entire sequence is already sorted.\\

But when testing synthetic , it is not proven if there is a position which performs better for certain cases than selecting the middle element as pivot on partition based algorithms (given the same constraints as the previous paragraph), nor if this bias for pivot selection can be used to tune the algorithm beforehand for certain cases.\\

\subsubsection{Clocked routines}
Not all parts of the program are subjected to monitoring via snapshots, as this approach is both nonsensical and non practical. We just change the execution of the program on certain points of the execution in order to gather metrics or to push elements to the log array. \\

Currently defined sections to be used as snapshot points are shown in the table on Table~\ref{TABLE:POINTS}:

\begin{table}[!ht]
    \centering
    \begin{tabularx}{\linewidth}{|r|r|X|}
        \hline
        Program flag & code & Description \\
        \hline
        \texttt{EXTRACTION\_STAGE\_BEGIN} & \texttt{10} & Start of the minima extraction\\
        \hline
        \texttt{EXTRACTION\_STAGE\_END} & \texttt{20} & End of the minima extraction \\
        \hline
        \texttt{ITERATION\_STAGE\_BEGIN} & \texttt{30} & Begin of a IQS or IIQS iteration \\
        \hline
        \texttt{ITERATION\_STAGE\_LOOP} & \texttt{40} & Middle point of a IQS or IIQS iteration \\
        \hline
        \texttt{ITERATION\_STAGE\_INTROSPECT} & \texttt{41} & Introspect stage of IIQS \\
        \hline
        \texttt{ITERATION\_STAGE\_END} & \texttt{50} & End of IQS or IIQS iteration \\
        \hline
        \texttt{PARTITION\_STAGE\_BEGIN} & \texttt{60} & Start of partitioning stage \\
        \hline
        \texttt{PARTITION\_STAGE\_END} & \texttt{70} & End of partitioning stage \\
        \hline
    \end{tabularx}
    
    \caption{Timed sections}
    \label{TABLE:POINTS}
\end{table}

\FloatBarrier


% \section{Hiearchies of algorithm design}
% \section{Experimental process}

\chapter{Methodology}
\chapterimage[height=3cm]{./fragments/memes/thor_tools}
\label{CHAPTER:METHODLOGY}

In this chapter we revisit the methodology explained in Section~\ref{SECTION:EXPERIMENTAL_ALGORITHMICS} in order to understand better how it is applied to our problem at hand.


\section{Rationale}
IncrementalQuickSort and its introspective version, IntrospectiveQuickSort, already have their theoretical analyses for the worst case instances. But such theoretical analysis is not always feasible, sometimes not easy and most of the time it is not realistic. As a practical example of it, when testing IQS against HeapSort for a full array sorting under architectures with small cache memory, IQS outperforms vastly HeapSort as it trashes the cache on each iteration. But when cache units are large enough to support the entire array, there is no point on using IQS, as most operations are actually solved on cache directly. There is a huge gap when it comes to practice on algorithm design, and IIQS is also not free of such problems.\\

The main issue arose when it comes to the analysis of the median-of-medians effects on the partition. As the execution of this algorithm offsets itself on each partition, it is the equivalent to run continuously a process which reduces the overall disorder of the sequence\footnote{We do not talk explicitly of any disorder metric seen in Section~\ref{SEC:MEASURING_DISORDER} as the effect depends on the process behind each execution. In this sense, we want our definition to be abstract and not to be tied with any algorithm implementation.}.\\

This effect displaces the elements on the sequence towards their expected position on it, due to the adaptive-sorting nature of both IQS and IIQS makes the overall running time dependent on the element distribution. This makes really hard the use of standard techniques like amortized analysis to study the behavior of IIQS. Even worse, due to the increased complexity of the algorithm, its theoretical analysis is likely to differ from the practical results.\\

The problem at hand is to modify the current implementation of IQS and IIQS to support an extra case, which is when the sequence of elements can have repeated elements on it. This is now a problem for many reasons as it messes up the pivot selection heuristics and partition stages of the original algorithm.\\

Due to the aforementioned reasons, we want to take an experimental approach to analyze this new instance and use the results to guide the development of an extension of this algorithm.\\

\FloatBarrier
\section{Methodology foundationals}
\label{SECTION:METHODOLOGY_FOUNDATIONALS}

\subsection{Algorithm instantiation hierarchy}
The main goal of this work is to devise if we can design version of both IQS and IIQS which can avoid the worst case when dealing with sequences that hold repeated elements. Once this goal is accomplished, then we need to study its behaviour in order to check ways to deliver the same performance for both repeating and non repeating sequences. Thus our instantiation hierarchy is as follows:\\

\subsubsection{Metaheuristics and algorithm paradigms}
IncrementalQuickSelect and IntrospectiveIncrementalQuickSelect are algorithms used for partial sorting, belonging to the \textit{partition-based adaptive sorting} family which are our optimization target using repeated elements on a sequence.\\

\subsubsection{Algorithms}
As both IQS and IIQS are partition-based algorithms, they both share common elements and routines, namely:\\

\begin{itemize}
    \item \textbf{Next}:This is the main process on both IQS and IIQS. It performs a minima extraction. It expected average running time is $O(n + log_2{n})$ on which $n$ is the size of the sequence being passed as input.
    \item \textbf{Partition}: Partitions a given sequence into three subsequences $p_1$, $p_2$ and $p_3$ which follows that $\forall~p_i \in p_1,~\forall~p_j \in p_2:~ p_i < p_j$ and $\forall~p_i \in p_3,~\forall~p_j \in p_2:~p_i < p_j$. It expected average running time is $O(m)$ on which $m$ is the size of the sequence being passed as input following $m \leq n$ on which $n$ is the total sequence length.
    \item \textbf{Swap}: Swaps two elements in the sequence in-place. It expected average running time is $O(1)$.
    \item \textbf{PushStack}: Pushes an element into the stack. It expected average running time is $O(1)$.
    \item \textbf{PullStack}: Pulls an element from the stack. It expected average running time is $O(1)$.
\end{itemize}

As for IIQS exclusive use routines we can mention:
\begin{itemize}
    \item \textbf{BFPRT:} Implementation of median of medians algorithm~\cite{Blum_Floyd_Pratt_Rivest_Tarjan_1973}. This algorithm is used as a fallback option for the random selection pivot selection performed during each iteration of IQS. It expected average running time is $O(m)$ on which $m \leq n$ is the size of the sequence being passed as input.
    \item \textbf{Median:} Sorts in-place an array of fixed size\footnote{Whilst on literature a median of medians of five elements is suggested, we do not want to tie the implementation of the algorithm to a fixed value, but rather become this size a parameter of our algorithm.} and then retrieves the element in the middle position. It expected average running time is $O(1)$, despite the complexity of the sorting mechanism used as the time used is constant and is not in function of the sequence length.
\end{itemize}

\subsubsection{Source program}
As for the implementation, our language of choice was C++ in conjuntion with Boost libraries for argument parsing. \\

\subsubsection{Object code}
Object code is obtained via direct compilation of the main source file. Due to portability concerns, this process is triggered via Makefiles but no special or separate treatment is done for the compilation artifacts.\\

\subsubsection{Process}
All experiments are executed on userspace without any extra execution privileges.\\

\subsection{Algorithm design hierarchy}

\subsubsection{System structure}
Experiments are structured in a way that prioritizes execution flexibility over convention. Initial versions of the implemented algorithm were developed using C, but this language as is has some problems when it comes to understanding from the developer's standpoint. In this aspect, C++ (which can be seen as an class-oriented version of C) helps to perform an easier modularizaton of the code, while maintaining the macro flexibility of C. Apart from the main algorithm implementations, we can identify the following structures:\\

\subsubsection{Main driver}
The entrypoint of our compilation unit is our daily driver, which takes charge of argument parsing, main execution and high-level operations as snapshot dumping.  Runtime options are parsed with the help of Boost's \texttt{boost::program\_options} library, which takes charge of parsing and converting arguments to their corresponding types. \\

The list of available arguments to pass to the main driver are shown in the table shown in Table~\ref{TABLE:ARGUMENTS}:\\


\begin{table}[!ht]
    \centering

    \begin{tabularx}{\linewidth}{|r|r|X|}
        \hline
        STD type & CLI option & Description \\
        \hline
        \texttt{bool} & \texttt{--log\_pivot\_time} & Enables clock for pivot selection runtime\\
        \hline
        \texttt{bool} & \texttt{--log\_iteration\_time} & Enables clock for iteration runtime \\
        \hline
        \texttt{bool} & \texttt{--log\_extraction\_time} & Enables clock for element extraction runtime\\
        \hline
        \texttt{bool} & \texttt{--log\_swaps} & Enables logging of swap information \\
        \hline
        \texttt{bool} & \texttt{--use\_bfprt} & Enables median of medians instead of random selection (only for IIQS) \\
        \hline
        \texttt{bool} & \texttt{--use\_iiqs} & Enables use of IIQS instead of IQS \\
        \hline
        \texttt{bool} & \texttt{--use\_random\_pivot} & Enables random pivot seletion \\
        \hline
        \texttt{bool} & \texttt{--enable\_reuse} & Enables pivot reuse \\
        \hline
        \texttt{double} & \texttt{--alpha\_value} & Sets $\alpha$ value \\
        \hline
        \texttt{double} & \texttt{--beta\_value} & Sets $\beta$ value \\
        \hline
        \texttt{double} & \texttt{--pivot\_bias} & Sets pivot selection bias \\
        \hline
        \texttt{int} & \texttt{--random\_seed\_value} & Sets random seed value \\
        \hline
        \texttt{std::size\_t} & \texttt{--input\_size} & Experiment input size \\
        \hline
        \texttt{std::size\_t} & \texttt{--extractions} & Experiment extractions to perform \\
        \hline
        \texttt{std::string} & \texttt{--input\_file\_value} & Input file path \\
        \hline
        \texttt{std::string} & \texttt{--output\_file\_value} & Output file path \\
        \hline    
    \end{tabularx}
    \caption{Arguments for main driver program}
    \label{TABLE:ARGUMENTS}
\end{table}

\subsubsection{Snapshot spec}
The information about the program execution is stored on \textit{snapshots}, which are inspired on Valgrind profiler stats. Snapshot contains a rather large amount of information, from program configurations, options, and current state. In order to capture snapshots, a set of precompiler macros has been defined to enable timing of whole routines, partial sections, and conditional values. \\

In order to minimize the performance hit, there is only one instance of \texttt{snapshot\_t} being kept on memory at all times which is passed as reference to the instances of IQS and IIQS. This snapshot is only saved on the record array log on RAM at certain keypoints. This record array log is preinitialized before the execution in order to avoid allocation calls during the execution phase. Currently the time unit used for benchmarking is \texttt{std::chrono::nanoseconds}, defined on the \texttt{TIME\_UNIT} macro.\\

The table on Table~\ref{TABLE:SNAPSHOT_STRUCTURE} shows the logged metrics described on the \texttt{snapshot\_t} structure:\\

\begin{table}[!ht]
    \centering
    \begin{tabularx}{\linewidth}{|r|X|}%|X|}
    \hline
    STD type & logged variable \\ % & Description \\ %& \\
    \hline
    \texttt{TIME\_UNIT} & \texttt{iteration\_time} \\ %& \\
    \hline
    \texttt{TIME\_UNIT} & \texttt{total\_iteration\_time} \\ %& \\
    \hline
    \texttt{TIME\_UNIT} & \texttt{partition\_time} \\ %& \\
    \hline
    \texttt{TIME\_UNIT} & \texttt{total\_partition\_time} \\ %& \\
    \hline
    \texttt{TIME\_UNIT} & \texttt{bfprt\_partition\_time} \\ %& \\
    \hline
    \texttt{TIME\_UNIT} & \texttt{total\_bfprt\_partition\_time} \\ %& \\
    \hline
    \texttt{TIME\_UNIT} & \texttt{extraction\_time} \\ %& \\
    \hline
    \texttt{TIME\_UNIT} & \texttt{total\_extraction\_time} \\ %& \\
    \hline
    \texttt{size\_t} & \texttt{current\_extraction\_executed\_partitions} \\ %& \\
    \hline
    \texttt{size\_t} & \texttt{total\_executed\_partitions} \\ %& \\
    \hline
    \texttt{size\_t} & \texttt{current\_iteration\_partition\_swaps} \\ %& \\
    \hline
    \texttt{size\_t} & \texttt{total\_executed\_partition\_swaps} \\ %& \\
    \hline
    \texttt{double} & \texttt{current\_iteration\_longest\_partition\_swap} \\ %& \\
    \hline
    \texttt{double} & \texttt{total\_executed\_longest\_partition\_swap} \\ %& \\
    \hline
    \texttt{size\_t} & \texttt{current\_iteration\_executed\_bfprt\_partitions} \\ %& \\
    \hline
    \texttt{size\_t} & \texttt{total\_executed\_bfprt\_partitions} \\ %& \\
    \hline
    \texttt{size\_t} & \texttt{current\_iteration\_bfprt\_partition\_swaps} \\ %& \\
    \hline
    \texttt{size\_t} & \texttt{total\_executed\_bfprt\_partition\_swaps} \\ %& \\
    \hline
    \texttt{double} & \texttt{current\_iteration\_longest\_bfprt\_partition\_swap} \\ %& \\
    \hline
    \texttt{double} & \texttt{total\_executed\_longest\_bfprt\_partition\_swap} \\ %& \\
    \hline
    \texttt{double} & \texttt{current\_extracted\_pivot} \\ %& \\
    \hline
    \texttt{size\_t} & \texttt{current\_stack\_size} \\ %& \\
    \hline
    \texttt{size\_t} & \texttt{total\_pushed\_pivots} \\ %& \\
    \hline
    \texttt{size\_t} & \texttt{total\_pulled\_pivots} \\ %& \\
    \hline
    \texttt{size\_t} & \texttt{current\_iteration\_pushed\_pivots} \\ %& \\
    \hline
    \texttt{size\_t} & \texttt{current\_iteration\_pulled\_pivots} \\ %& \\
    \hline
    \texttt{size\_t} & \texttt{current\_extraction} \\ %& \\
    \hline
    \texttt{size\_t} & \texttt{input\_size} \\ %& \\
    \hline
    \texttt{char} & \texttt{snapshot\_point} \\ %& \\
    \hline
\end{tabularx}
\caption{Snapshot structure}
\label{TABLE:SNAPSHOT_STRUCTURE}
\end{table}


\subsubsection{Algorithm and data structure design}
Testbench implementation is divided into four major components:\\

\begin{itemize}
    \item{
        \textbf{IQS C++ Implementation}: Base C++ implementation of IQS with support for C++ STD container classes. One of the differences with the standard implementation of IQS is the presence of \texttt{IQS::random\_between} and \texttt{IQS::biased\_between} methods, which allows control over the pivot selection methods.
    }
    \item{
        \textbf{IIQS C++ Implementation}: Base C++ implementation of IIQS with support for C++ STD container classes. Inherits all components for IQS so this implementation only overloads \texttt{IQS::next} method and adds \texttt{IIQS::bfprt}, intended to support the extra operations needed by IIQS.
    }
    \item{
        \textbf{IQS low-level C++ Implementation}: C++ implementation of IQS without support for C++ STD container classes, relying only on direct memory allocation. This implementation was not benchmarked as it is only intended to be used as reference.
    }
    \item{
        \textbf{IIQS low-level C++ Implementation}: C++ implementation of IIQS without support for C++ STD container classes, relying only on direct memory allocation. This implementation was not benchmarked as it is only intended to be used as reference. All methods from low-level IQS are inherited here.
    }
\end{itemize}

\subsubsection{Implementation and algorithm tuning}
On the original IIQS analysis, randomized sequences and sorted sequences were used as tests. The original problem constrained all worst case input instances to be ordered sequencies in order to ease understanding. On ordered sequences is easier to see when a pivot selection fails by misusing the stack, thus not reducing the problem size. But since we now are dealing with repeated elements, new sequences for input are needed to test such cases.\\

\begin{itemize}
        
    \item{\textbf{Randomized sequences}: 
    This is our classical test case, on which all the elements are shuffled without any special criteria.}

    \item{\textbf{Ascending sequences}: 
    Used to generate a synthetic worst-case instance for IQS, this sequence is ordered in ascending order.}

    \item{\textbf{Descending sequences}: 
    Used to generate a synthetic worst-case instance for IQS, this sequence is ordered in descending order.}

    \item{\textbf{Constrained classes}: 
    Given $m < n$ the number of classes on the sequence, we want to test the effect of the ratio $\frac{m}{n}$ for a fixed number classes to devise if there is a relationship between the number of classes and the running time of the algorithm. This input is shuffled after its generation.}

    \item{\textbf{Constrained classes with random noise}: 
    In addition to the previous instance, we also add a random number of elements which do not belong to any instances of $m$ to induce random noise on the sample. This input is shuffled after its generation.}

    \item{\textbf{Shuffled sequences with sorted segments}: 
    Based on a mix of \textit{Runs}, \textit{SUS} and \textit{SMS.SUS} metrics for adaptive sorting, this input attempts to test the effect of presortedness on the execution of IQS and IIQS. To generate this input we first generate a shuffled input and then for each subsegment of the shuffled sequence we execute a partial sorting.}

    \item{\textbf{Randomized sequence with ignored noise}: 
    This input is intended to test if discontinuities on the sorting process can affect the performance by ignoring certain swaps. To accomplish this, we take a randomized sequence and then for a given amount of elements on the sequence we put the value that belong to their position.}
\end{itemize}

Aditionally, due to the nature of the problem, we have decided to constrain the following two aspects of the implementation in order to ensure performance and replication of results. So, they can be peer-validated at a later stage. Replication is achieved by taking a monte-carlo simulation~\cite{10.5555/1614191} approach using the following means:\\

\begin{itemize}
    \item{\textbf{Fixed seeding}
    For all experiments, all inputs are generated beforehand on the same instance of the machine in sequential order and providing the same seed for all random number generators.}
    
    \item{\textbf{Systematic randomization}
    For all processes that require randomization, the random values provided are extracted from a separate file beforehand, this ensures that all extractions of random numbers for the use of the algorithm are delivered in the same order across executions.}
\end{itemize}

\subsubsection{Code tuning}
%only one instance
%preinstanciation before clocking
\begin{itemize}
    \item{\textbf{Unique snapshot intance}: 
    In order to minmize memory consumption and allocation operations only one snapshot instance is initialized for a whole experiment, and it is passed as reference along the whole program.}
    \item{\textbf{Memory pre-allocation}: 
    All test cases, files, snapshot space, and random generated number are computed by an external process and they are fed into the program via file inputs which are read using STL \texttt{std::ifstream} and initialized before all tests start, so memory is allocated already at this point in order to prevent reallocation operations.}
    \item{\textbf{Unique source of truth}:
    All random numbers used along implementations are extracted from a unique source, from the same allocated space during runtime. All alocations are performed before the main execution and clocks start running in order to ensure that inicialization process does not affect runtimes due to memory allocation overhead.}
\end{itemize}

\subsubsection{System software}
Our compiler is GNU GCC 9.3.0 configured for a x86\_64-linux-gnu target with posix thread modeling. All compiler optimizations are disabled in order to track time more acurrately.
\subsubsection{Platform and hardware}
The specs of the machine used are shown in the table on Table~\ref{TABLE:SPECS}:\\

\begin{table}[!ht]
    \centering
    \begin{tabularx}{\linewidth}{|l|X|}
        \hline
        Item & Product ID \\
        \hline
        Processor  & Ryzen 5 Series 3600, 6C/12T 3.6 GHz base processor clock 4.2 GHz max boost, 35MB cache, unlocked clock settings \\
        \hline
        Memory  & Team T-FORCE DARK Za 32GB (2 x 16GB) DDR4 3600 MHz (PC4 28800) TDZAD432G3600HC18JDC01, dual-channel enabled, XMP profile 1 enabled\\
        \hline
        Storage  & WD M.2 SSD 480GB WDS480G2G0B \\
        \hline
        Motherboard  & MSI B450M PRO-VDH MAX, AM4 \\
        \hline
        Power Suppy Units  & EVGA 600W W1, 80+ Certified\\
        \hline
        Video Adapter  &  Galax Video NVIDIA GeForce GTX1650 1-Click OC  \\
        \hline
        Operating System  &  Pop!\_OS 20.04 LTS  \\
        \hline
        Kernel  &  Linux rspk-shoukugun 5.4.0-7629-generic \#33\~{}1589834512\~{}20.04\~{}ff6e79e-Ubuntu SMP Mon May 18 23:29:32 UTC  x86\_64 x86\_64 x86\_64 GNU/Linux\\
        \hline
        Linux Version  & Linux version 5.4.0-7629-generic (buildd@lcy01-amd64-013) (gcc version 9.3.0 (Ubuntu 9.3.0-10ubuntu2)) \#33\~{}1589834512\~{}20.04\~{}ff6e79e-Ubuntu SMP Mon May 18 23:29:32 UTC \\ 
        \hline
    \end{tabularx}    
    \caption{Test machine specs}
    \label{TABLE:SPECS}
\end{table}
\FloatBarrier
\section{Experimental process breakdown}
% first generate inputs
% perform gridsearch
% analyze results
% present
% raise next question
Now, we present a breakdown of the experimental process definitions needed in order to begin executing our experiments.\\

\subsection{Experimental cycle}
As mentioned before in Section~\ref{SUBSECTION:EXPERIMENTAL_ALGORITMICS_CONSIDERATIONS}, this process is being handled in a cyclic manner, and each experiment is controlled by their own Jupyter Notebook. Now we describe the steps followed through the realization of the experiments of this report.\\

\subsubsection{Hypothesis}
We begin by raising a question and a doable answer to it which we want to prove if it holds for the current experimentation cycle. In the context of this report, hypothesis are code tuning improvements that we want to check before building a solution for our problem.\\

\subsubsection{Input and execution setup}
After defining our topic, we start generating inputs and planning the experiment execution. As for the inputs, we already defined in Section~\ref{SECTION:METHODOLOGY_FOUNDATIONALS} how the inputs are be generated, as for each experiment inputs are to be selected accordingly on what we want to test or explore. \\

On the other hand, as the main driver already accepts a defined set of parameters to control the execution, those are used to setup the experiment environment. From now on, we refer to the cross product of the combination of inputs and program parameters as the \textit{search space}.\\

\subsubsection{Execution}
At this point we execute a GridSearch over out \textit{search space} in sequential order.This way we reduce the amount of disturbances that the experiment can suffer. After the execution of the experiments, snapshots are stored on a single ASCII file using a comma separated schema, which is is later used for the analysis.\\

\subsubsection{Analysis}
The results are gathered on Jupyter and examinated in order to get insight on the phenomena and to check if the hypothesis is valid or not. After we gather enough information, a discussion on the results is held, which is written together with the results on this report.\\

\subsection{Metrics and indicators}
In order to evaluate the experiments, we need to define beforehand some metrics to determine which aspects of the execution are evaluated during the experiments. The difference between comparing raw data and use metrics is the pre-processing being made in order to gain useful insight about what is going on under the hood before executing our first benchmark.\\

\subsubsection{Swaps}
Using the definitions for measuring disorder given in Section \ref{SEC:MEASURING_DISORDER}, we can establish metric for complete sorting algorithms. Both IQS and IIQS falls into the definition of incremental sorting as shown on Section~\ref{SEC:INCREMENTAL_SORTING}, making its use natural to us. But as our study consists on analysing the behaviour of extractions and not of a complete sorting execution, such metrics are unrealistic to such purposes and unnatural.\\

Still, such definitions can help us to establish a base to define our own disorder metric given the following known premises:\\

\begin{itemize}
    \item Both IQS and IIQS rely on \textit{partition} in order to perform the partial sorting the same way as \textit{QuickSort}.
    \item QuickSort, being a adaptive sorting algorithm, is influenced by presortedness.
    \item \textit{Dis}, \textit{Max}, \textit{Exc}, \textit{Rem}, \textit{Runs}, \textit{SUS}, and \textit{SMS.SUS} are applicable metrics for \textit{QuickSort}.
    \item Both IQS and IIQS perform their heavy lifting at the partition stage.
\end{itemize}

Then, the minimum common denominator of the aforementioned premises is the behaviour of swap operations. Given the nature of \textit{partition} operation, it is expected to not exchange any elements (\textit{Exc}) on a sorted sequence, and transitively, each segment of every iteration of IQS must follow the same property given that it is being performed in-place.\\

On the other hand, as it can be seen on the executions for worst case of IQS, as there are long ascending sub-sequences on the input (\textit{Runs}, \textit{SUS}, \textit{SMS.SUS}), the fastest the partition stage ends, as it is not performing any swaps and in some cases not storing any pivots at all.\\

Now when a swap operation occurs at the partition stage, the elements being swapped and the pivot share a partial sorting relationship between them. Given that this relationship can occur at any point of the sequence we can state the following aspects of the swap operation:\\

\begin{itemize}
    \item The longest the distance of the swap being performed from the pivot, the elements are more far away from their actual position (\textit{Dis}, \textit{Max}).
    \item A sorted sequence does not perform any swaps on their partition stage.
    \item Given that all elements are being sorted in-place, and partitions are executed in a recursive way respect the pivot position, the previous two properties are transitive respect the partitions that are being generated during the extraction of a minima.
\end{itemize}

Then, we establish the following two disorder metrics for IQS and IIQS:\\

\begin{itemize}
    \item \textbf{N\_SWAPS}: As the number of swaps performed for a given iteration of IQS. This metric can be extended as cumulative regarding the extraction of a minima. The values for \textbf{N\_SWAPS} are in the range $[0, n^2]$ for a given iteration.
    \item \textbf{MAX\_SWAP}: As the longest swap performed during a iteration of IQS. This metric only applies to each individual execution of partition, as the maximum distance is bound to decrease on each iteration. The values for \textbf{MAX\_SWAP} are in the range $[0, m]$, on which $m$ is the size of the current partition which follows $m \leq n$.
\end{itemize}

\subsubsection{Stack operations}
A key aspect of IQS and IIQS is the amount of extra memory needed to perform the sorting. Under normal conditions, IQS requires $log_2(n)$ extra space in order to maintain all pivots for the first extraction. This makes the first extraction the most time-consuming operation as it partitions over $n$ elements and pushes $log_2(n)$ pivots into the stack in average.\\

One of the problems of IQS was the fact that in certain cases, $n$ pivots get stored on the stack, forcing for the non-introspective version to fix the stack size at $n$, whilst the introspective version due to its most stable behaviour can be safely set at $log_{1.7}(n)$ thanks to the median-of-median algorithm effect.\\

In light of the aforementioned facts, it makes sense that any modification being made to IQS or IIQS must also compare the performance of the stack growth due to caching and memory consumption concerns. In contrast to the previous work on IIQS, this version of the analysis also considers other extra metrics such as number of pulled elements and number of pushed elements per iteration and per minima extraction, as their behaviour is directly connected to the time used by the partitioning stage.\\

\subsubsection{Number of executed subroutines}
In line with the stack problem, it is sane to establish if the number of elements in the stack is in direct relation to the executions of the partition routine, and in the same way, the executions of the partition routine has direct relationship with the number of elements extracted. In this regard, we also want to log the time that both routines get called during minima extraction to study if they had any effect on the total running time of IQS and IIQS.\\

\subsubsection{Pivot bias}
The preferred way of testing worst-case executions on partition-based sorting algorithms is to fix the pivot selection to a position which ensures the worst outcome each time. This is the main reason on why introduction of randomization for selection for pivots is so effective on maintaining expected average case on such algorithms. For IQS, worst case comes from choosing the lowest or the highest element on the sequence which can be accomplished by fixing the pivot position on the edges of the sequence to be partitioned, given that the entire sequence is already sorted.\\

But when testing synthetic , it is not proven if there is a position which performs better for certain cases than selecting the middle element as pivot on partition based algorithms (given the same constraints as the previous paragraph), nor if this bias for pivot selection can be used to tune the algorithm beforehand for certain cases.\\

\subsubsection{Clocked routines}
Not all parts of the program are subjected to monitoring via snapshots, as this approach is both nonsensical and non practical. We just change the execution of the program on certain points of the execution in order to gather metrics or to push elements to the log array. \\

Currently defined sections to be used as snapshot points are shown in the table on Table~\ref{TABLE:POINTS}:

\begin{table}[!ht]
    \centering
    \begin{tabularx}{\linewidth}{|r|r|X|}
        \hline
        Program flag & code & Description \\
        \hline
        \texttt{EXTRACTION\_STAGE\_BEGIN} & \texttt{10} & Start of the minima extraction\\
        \hline
        \texttt{EXTRACTION\_STAGE\_END} & \texttt{20} & End of the minima extraction \\
        \hline
        \texttt{ITERATION\_STAGE\_BEGIN} & \texttt{30} & Begin of a IQS or IIQS iteration \\
        \hline
        \texttt{ITERATION\_STAGE\_LOOP} & \texttt{40} & Middle point of a IQS or IIQS iteration \\
        \hline
        \texttt{ITERATION\_STAGE\_INTROSPECT} & \texttt{41} & Introspect stage of IIQS \\
        \hline
        \texttt{ITERATION\_STAGE\_END} & \texttt{50} & End of IQS or IIQS iteration \\
        \hline
        \texttt{PARTITION\_STAGE\_BEGIN} & \texttt{60} & Start of partitioning stage \\
        \hline
        \texttt{PARTITION\_STAGE\_END} & \texttt{70} & End of partitioning stage \\
        \hline
    \end{tabularx}
    
    \caption{Timed sections}
    \label{TABLE:POINTS}
\end{table}

\FloatBarrier


\chapter{Summary}
% \chapter{Experiments}
% \section{Experimental setup}
% \section{Experimental results}
% \section{Metrics and indicators}
% \subsection{Local entropy decay}
% \subsection{BFPRT executions}
% \subsection{Partitioner executions}

% \section{Tunning}
% \subsection{Data generation and execution control}


% %% contenido del segundo capítulo
% \chapter{Segundo Capítulo}
% Sólo para probar algunas cosas como las referencias.
% La primera cita es a Lamport~\cite{lamport79}.
% La segunda cita es para Lamport nuevamente~\cite{lamport78}.
% La última cita es para Keleher \emph{et al.}~\cite{keleher92}.


% %% contenido del tercer capítulo
% \chapter{Tercer Capítulo}
% Sólo para incluir figuras y tablas.
% \begin{figure}[h]
%   \vspace*{1cm}
%   \includegraphics[bb=0 0 640 480, width=.5\linewidth]{latexlogo.png}
%   \vspace*{1cm}
%   \caption{La primera figura de la memoria}
% \end{figure}
% \begin{table}[h]
%   \vspace*{1cm}
%   (aqui debiera ir la tabla)
%   \vspace*{1cm}
%   \caption{La primera tabla de la memoria}
% \end{table}


%% ambiente glosario
\begin{glosario}
  \item[El primer término:] Este es el significado del primer término, realmente no se bien lo que significa pero podría haberlo averiguado si hubiese tenido un poco mas de tiempo.
  \item[El segundo término:] Este si se lo que significa pero me da lata escribirlo...
\end{glosario}


%% genera las referencias
\bibliography{refs}


%% comienzo de la parte de anexos
\appendixpart

%% contenido del primer anexo
\appendix{El Primer Anexo}
Aquí va el texto del primer anexo...

\section{La primera sección del primer anexo}
Aquí va el texto de la primera sección del primer anexo...

\section{La segunda sección del primer anexo}
Aquí va el texto de la segunda sección del primer anexo...

\subsection{La primera subsección de la segunda sección del primer anexo}


%% contenido del segundo anexo
\appendix{El segundo Anexo}
Aquí va el texto del segundo anexo...

\section{La primera sección del segundo anexo}
Aquí va el texto de la primera sección del segundo anexo...

%% fin
\end{document}



