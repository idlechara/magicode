\chapter{Workhorse experiment}
\chapterimage[height=3cm]{./fragments/memes/2010_and_2020}
\label{CHAPTER:WORKHORSE_EXPERIMENT}

In this section, we present the consolidation of our previous experiments in the form of a bigger experiment with a more in-depth analysis.\\

\section{Repeating elements problem}
\label{SECTION:STACK_STRUCTURE}

Contrary to our initial expectations, BFPRT parameters did not influence the result when dealing with a sequence with repeated elements. So, for this experiment we change tune a different level, as we now see the structures involved on its resolution.\\

Reminiscing the main concept of incremental sorting, this approach is designed to tackle the problem of extracting elements from a sequence in a sorted fashion in optimal time, when there is no information beforehand on how many elements are needed to be extracted from its source.\\

In all our previous experiments we always took as granted the definition of our input sequence, which from now we denote as $S_{in}$. For all our implementation and structural concerns, $S_{in}$  have a defined size $\norm{S_{in}}$ over a $\mathcal{Z}^{\geq0}$ domain.\footnote{This may seem obvious, but as this structured can be implemented in an arbitrating way, we need to constrain our restrictions for the program as much as possible in order to clearly define our algorithm bounds.} 

Depending on the problem to be solved using IQS/IIQS, we may need different representations for the output. In this regard, we can see IQS and IIQS as a function which maps a sequence $S_{in}$ into another sequence $S_{out}$ which contains the same elements but binned via a certain criterion. We do not specify the criteria at this point, as it is not the point of this work to compare different comparison mechanisms. For such effects, it is safe to assume that our comparison criteria is the one in Section~\ref{SEC:INCREMENTAL_SORTING}.\\

\subsection{Current paradigm}
In rough terms, both IQS and IIQS algorithms belong to the divide-and-conquer paradigm as they recursively break our problem into a smaller one which are solved later. Contrary to a complete sorting algorithm, on partial sorting we do not expect to solve the problem for our entire input, in other words, we do not need to solve all the recursion steps in the execution tree in order to get an accepted output. \\

As an example of this let us consider a sequence $T={4,1,2,3}$. Now let us say that we need a partial sorting over the first two elements of $T$. Then our solutions are $T_1={1,2,3,4}$ and $T_2={1,2,4,3}$. Both $T_1$ and $T_2$ are accepted outputs as the first two elements are sorted. This aforementioned definition does not bound our final implementation to solve all the recursion tree induced by our paradigm and considers this solution as a state of our recursion tree. \\

Another ---less graphical--- way of depicting this situation is by modeling our execution tree as a non-deterministic stacked state machine, on which our root node as a final state, and the leaves of the recursion tree as initial states ---the elements that can be placed in their correct position--- with strategically placed elements to be consumed in the stack. There is no need to go through all the states as long as we reach an accepted final state.\\


In practice\footnote{As another way to say, in common implementations.}, few sorting algorithms run pre-checks in order to avoid certain cases. As these \emph{sanity checks} are bound to cost at least $O(n)$ time, they are not feasible in most cases. That is why introspective algorithms like introsort~\cite{MusserIntroSort} and IIQS~\cite{7416566} integrate such process to be part of already existing processes, in order to not affect the overall complexity. \\

\subsection{Paradigm shift}

Sequences with repeating elements are a different story. Let us take as example a sequence of elements $T'={1,1,1,1}$, which shares the same size as our previously introduced $T$ sequence, but this one has only one class of elements. Extrapolating the definition of a sequence with only one class among all sequence elements stated in Section~\ref{SECTION:NOTE_ON_MEDIAN_SELECTION}, this sequence is already sorted as it is. As all the possible permutations on $T'$ share the same values, and the comparison are being made on the values itself, then there is no problem to solve. As our minimal problem has the same size of our original one, our paradigm is no longer valid anymore.\footnote{In fact, the invalidation of the problem paradigm is the real reason behind IIQS failure.}\\

If we want to continue using IQS or IIQS to solve our incremental sorting problem, we first need to validate our paradigm and then solve the problem at hand. Paraphrasing this statement, we need to integrate a process which allows us to map our input to an accepted one without affecting our current complexity.\\

\subsection{Counting elements to reduce complexity}

As shown in Section~\ref{SUBSECTION:BASE_BENCHMARK}, our ideal input for both IQS and IIQS is a sequence with unique elements. Counting sort already solves this problem by mapping our elements present in $S_{in}$ as a set $C$ of ordered pairs $(v,f)$ on which $v$ is the value in $S_{in}$, and $v$ is the frequency of the value in $S_in$.~\cite{10.5555/280635} Sorting is achieved by the container of the set or by extending the set to a sequence of elements and storing them accordingly --- this being the preferred way of implementation. \\

Unfortunately, while counting sort performs in lineal time, this algorithm does not exhibit a reasonable space-complexity performance when compared to IIQS, as in the worst case it requires extra $O(n)$ space whereas IIQS requires $O(log_2(n))$.\\

\subsection{Mapping}
\label{SECTION:MAPPING}

As shown in Section~\ref{SECTION:NOTE_ON_MEDIAN_SELECTION}, our sorting problem does not take into account the position of the elements and only relies on the value of such elements. This allows us to use the mapping induced by counting sort, $C_{in}$, as an intermediate representation of our original sequence $S_{in}$.\\

Let us define $S_{repeated(\gamma)}$ as a sequence of elements $(s_0, s_1,...,s_{n-1},s_{n})$ on which $\forall ~s_i \in [0,n]:~s_i = \gamma$. Then, every sequence $S_{in}$ can be seen as a concatenation of $m$ sequences of size $k$ on which $k \leq m$. Finally, we can establish two functions $roll$ and $unroll$ defined as follows:

\begin{align*}
    roll \colon S_{repeated(\gamma)} &\to C_{\gamma}\\
    (s_0, s_1,...,s_{n-1},s_{n})  &\mapsto (n).
\end{align*}

And analogously:

\begin{align*}
    unroll \colon C_{\gamma} &\to S_{repeated(\gamma)}\\
    (n) &\mapsto (s_0, s_1,...,s_{n-1},s_{n}).
\end{align*}

This mapping allows us to represent any sequence of elements as a sequence of pairs which represent the element and how many repetitions of this element are concatenated to it.\footnote{This representation is also known as \emph{frequency table}.}\\

\subsection{Integrating counting strategies as an external process}
As both $roll$ and $unroll$ functions operate per each element and not at sequence level, we can extend their usage both inside of our current IQS and IIQS implementation or as an external process.\\

We denote our candidate sorting algorithm as an anonymous function $\lambda_{sort}$. Then, our external reduction strategy is as follows:


\begin{algorithm}
\caption{External reduction}\label{ALG:EXTERNAL_IQS}
\begin{algorithmic}[1]
    \Procedure{$external\_strategy$}{$S_{in}, k, \lambda_{sort}$}
    \State $C_{in} \gets roll(S_{in})$
    \State $S_{sorted}' \gets \lambda_{sort}(C_{in}, S, k)$
    \State $S_{out}' \gets [~]$
    \For{$s \in S_{sorted}'$}
        \State $S_{out}' \gets S_{out}'~^\frown~unroll(s, C_{in}(s))$
    \EndFor
    \State \Return $S_{out}$
    \EndProcedure
\end{algorithmic}
\end{algorithm}

Algorithm~\ref{ALG:EXTERNAL_IQS} shows an implementation of an external reduction strategy. We assume that roll operation, as it is an instance of counting sort takes $O(n)$ time and onwards also take $O(n)$. \\

As the average running time for IQS and IIQS is $O(n~+~k~log_2(k))$, then the addition of our external strategy to deal with repeating elements introduces a fixed $n + l$ overhead on which $l$ denotes the number of classes present on $S_{in}$. But this overhead does not change the complexity of both algorithms. However this does not prevent IQS from hitting its worst case scenario unless we control how the keys retrieved from $C_{in}$ are retrieved.\\

Another drawback of this implementation is that since we are using an external strategy to reduce the repeating elements problem, we need $l$ extra space. In the worst case, $l~=~\norm{S_{in}}$, which is suboptimal in relation to IQS.\\

\subsection{Integrating counting strategies as part of the partition process}

When integrating this problem reduction stage directly into the algorithm, then we need to materialize it at implementation level. For such effects, the ideal place to install it is on the stack which tracks the information for the next calls. \\

In order to accomplish this, we change the structure of the stack used by extending the operations involved on it, allowing to group a range of elements, instead of just a single element. For such effects, the stack store pairs of elements $u = (p_{start}, p_{end})$ on which $p_{start}$ is the first position of the pivot and $p_{end}$ is the last position on the array. \\

By extension of the concepts shown in Sections~\ref{SECTION:MAPPING} and~\ref{SECTION:NOTE_ON_MEDIAN_SELECTION}, we extend the operations $partition$, $top$, and $pop$ to support both $roll$ and $unroll$ definitions as follows:



\begin{algorithm}
    \caption{Binned Stack top}\label{ALG:STACK_TOP}
    \begin{algorithmic}[1]
        \Procedure{$Stack.binnedTop$}{$S$}
        \State $(p_{start}, p_{end}) \gets S.top()$
        
        \State \Return $p_{start}$
        \EndProcedure
\end{algorithmic}
\end{algorithm}


\begin{algorithm}
    \caption{Binned Stack pop}\label{ALG:STACK_POP}
    \begin{algorithmic}[1]
        \Procedure{$Stack.binnedTop$}{$S$}
        \State $(p_{start}, p_{end}) \gets S.top()$
        \State $S.pop()$
    
        \If{$p_{start} < p_{end}$}
            \State $S.push((p_{start}+1, p_{end}))$
        \EndIf
        \EndProcedure
\end{algorithmic}
\end{algorithm}

\begin{algorithm}
\caption{Binned Three-way Partition}\label{ALG:DUTCH_FLAG_PARTITION_RANGED}
\begin{algorithmic}[1]
    \Procedure{$binnedPartition$}{$A, p$}
    \State $k \gets \norm{A}$
    \State $i \gets 0$
    \State $j \gets 0$
    \While{$j < k$}
        \If{$A_j < p$}
            \State $swap(A_i, A_j)$
            \State $i \gets i+1$
            \State $j \gets j+1$
        \ElsIf{$A_j > p$}
            \State $k \gets k-1$
            \State $swap(A_j, A_k)$
        \Else
            \State $j \gets j+1$
        \EndIf
    \EndWhile
    \State \Return $(i,k)$
    \EndProcedure
\end{algorithmic}
\end{algorithm}

\section{Binned IIQS}
The extensions shown in Algorithms~\ref{ALG:STACK_TOP}, and~\ref{ALG:STACK_POP} fixes the position of the index returned from the middle section as pivot to the left-most element. As proven in Section~\ref{SECTION:IQS_PARAMS} this bias generates a synthetic best-case execution, which is used directly after the partition stage. Those two algorithms integrate both $roll$ and $unroll$ definitions by storing the range of pivots in a map notation.\\


On the other hand the extensions for Algorithm~\ref{ALG:DUTCH_FLAG_PARTITION_RANGED} are integrated into IIQS almost directly with no major modifications, allowing a transparent use without any extra overhead, as shown in Algorithm~\ref{ALG:RANGED_IIQS}.\\

\begin{algorithm}
  \begin{algorithmic}[1]
    \caption{Binned IIQS} \label{ALG:RANGED_IIQS}
    \Procedure{b-iiqs}{$A, S, k$}
    \While{$k < S.binnedTop()$}
        \State $pidx \gets random(k,S.binnedTop()-1)$
        \State $pidx, range \gets partition(A_{k,S.binnedTop()-1}, pidx)$
        \State $m \gets S.binnedTop() - k$
        \State $\alpha \gets 0.3$
        \State $\beta \gets 0.7$
        \State $idx_\alpha \gets k + \alpha m$
        \State $idx_\beta \gets k + \beta m$
        \State $r \gets -1$

        \If{$pidx > idx_\beta \wedge idx_\alpha < pidx $}
            \State $pidx \gets pick(A_{r+1,S.binnedTop()-1})$
            \State $pidx, range \gets binnedPartition(A_{r+1,S.binnedTop()-1},pidx)$
        \EndIf
        \State S.push($(pidx, range)$)

    \EndWhile
    \State S.pop()
    \State \textbf{return} $A_{k}$\label{RANGED_IIQS}
    \EndProcedure
  \end{algorithmic}
\end{algorithm}

As both Algorithms~\ref{ALG:STACK_TOP}, and~\ref{ALG:STACK_POP} have $O(1)$ worst-case complexity and Algorithm~\ref{ALG:DUTCH_FLAG_PARTITION_RANGED} maintains $O(n)$ complexity on which $n$ denotes the number of elements in the sequence to partition. Both expected and worst-case time complexity for this \emph{Binned IntrospectiveIncrementalQuickSort} (\emph{bIIQS} from now on) remains $O(n~+~k~log_2(k))$ and our original $O(log_2(k))$ space complexity is maintained as the stack only doubles its size at most.\\

As elements in the stack are stored as a pair, this enables the use of custom structures to store such pairs, which enables a better use of DRAM bursts on the target machine. This guarantees a nearly zero overhead on read and write operations for the stack.\\

The experiments developed on Sections~\ref{SECTION:IQS_PARAMS} also helps us to fix our parameters for the execution of the introspective steps to $\alpha=0.5$ and $\beta=0.65$ as experimentally has been proven to offer the best performance from all tested combinations.\\

This makes \emph{bIIQS} a direct replacement of \emph{IIQS} for all cases. Additionally, if worst-case execution is not relevant, the same modifications are available for its use on \emph{IQS} implementations in order to support repeated elements on its input.\\

\section{Experimental evaluation}

Now, we present the experimental results for our proposed implementation of bIIQS against our existing implementations of IQS and IIQS to validate our aforementioned proposal.\\


\subsection{Performance comparison for unique sequences}

Experiments have shown that bIIQS does not induce an increase of the algorithm complexity for our standard test cases such as random sequences, ascending sorted sequences and descending sorted sequences, as shown in detail in Figures~\ref{FIG:WORKHORSE_BENCHMARK_01},~\ref{FIG:WORKHORSE_BENCHMARK_02} and ~\ref{FIG:WORKHORSE_BENCHMARK_03}. Therefore, our hypothesis on the equivalence of IIQS and bIIQS for our base IQS test cases is confirmed. \\


\begin{figure}
    %FIG:WORKHORSE_BENCHMARK_01
    %01_basebenchmark_01_random_case

    \centering
    \begin{subfigure}[b]{\textwidth}
        \centering
        \includegraphics[width=0.35\textwidth]{./fragments/05_workhorse_experiment/images/01_basebenchmark_01_random_case.png.0_0.png}
        \includegraphics[width=0.35\textwidth]{./fragments/05_workhorse_experiment/images/01_basebenchmark_01_random_case.png.0_1.png}
        \caption{Cumulated extraction time}
        \label{FIG:WORKHORSE_BENCHMARK_01__0_0}
    \end{subfigure}

    \begin{subfigure}[b]{\textwidth}
        \centering
        \includegraphics[width=0.35\textwidth]{./fragments/05_workhorse_experiment/images/01_basebenchmark_01_random_case.png.1_0.png}
        \includegraphics[width=0.35\textwidth]{./fragments/05_workhorse_experiment/images/01_basebenchmark_01_random_case.png.1_1.png}
        \caption{Extraction time}
        \label{FIG:WORKHORSE_BENCHMARK_01__0_1}
    \end{subfigure}

    \begin{subfigure}[b]{\textwidth}
        \centering
        \includegraphics[width=0.35\textwidth]{./fragments/05_workhorse_experiment/images/01_basebenchmark_01_random_case.png.2_0.png}
        \includegraphics[width=0.35\textwidth]{./fragments/05_workhorse_experiment/images/01_basebenchmark_01_random_case.png.2_1.png}
        \caption{Stack size}
        \label{FIG:WORKHORSE_BENCHMARK_01__0_2}
    \end{subfigure}
    
    \caption{bIIQS benchmark for a random sequence with $100\times10^3$ elements.}
       \label{FIG:WORKHORSE_BENCHMARK_01}
\end{figure}


\begin{figure}
    %FIG:WORKHORSE_BENCHMARK_02
    %01_basebenchmark_02_sort_a_case

    \centering
    \begin{subfigure}[b]{\textwidth}
        \centering
        \includegraphics[width=0.35\textwidth]{./fragments/05_workhorse_experiment/images/01_basebenchmark_02_sort_a_case.png.0_0.png}
        \includegraphics[width=0.35\textwidth]{./fragments/05_workhorse_experiment/images/01_basebenchmark_02_sort_a_case.png.0_1.png}
        \caption{Cumulated extraction time}
        \label{FIG:WORKHORSE_BENCHMARK_02__0_0}
    \end{subfigure}

    \begin{subfigure}[b]{\textwidth}
        \centering
        \includegraphics[width=0.35\textwidth]{./fragments/05_workhorse_experiment/images/01_basebenchmark_02_sort_a_case.png.1_0.png}
        \includegraphics[width=0.35\textwidth]{./fragments/05_workhorse_experiment/images/01_basebenchmark_02_sort_a_case.png.1_1.png}
        \caption{Extraction time}
        \label{FIG:WORKHORSE_BENCHMARK_02__0_1}
    \end{subfigure}

    \begin{subfigure}[b]{\textwidth}
        \centering
        \includegraphics[width=0.35\textwidth]{./fragments/05_workhorse_experiment/images/01_basebenchmark_02_sort_a_case.png.2_0.png}
        \includegraphics[width=0.35\textwidth]{./fragments/05_workhorse_experiment/images/01_basebenchmark_02_sort_a_case.png.2_1.png}
        \caption{Stack size}
        \label{FIG:WORKHORSE_BENCHMARK_02__0_2}
    \end{subfigure}
    
    \caption{bIIQS benchmark for an ascending sorted sequence with $100\times10^3$ elements.}
       \label{FIG:WORKHORSE_BENCHMARK_02}
\end{figure}



\begin{figure}
    %FIG:WORKHORSE_BENCHMARK_03
    %01_basebenchmark_03_sort_d_case

    \centering
    \begin{subfigure}[b]{\textwidth}
        \centering
        \includegraphics[width=0.35\textwidth]{./fragments/05_workhorse_experiment/images/01_basebenchmark_03_sort_d_case.png.0_0.png}
        \includegraphics[width=0.35\textwidth]{./fragments/05_workhorse_experiment/images/01_basebenchmark_03_sort_d_case.png.0_1.png}
        \caption{Cumulated extraction time}
        \label{FIG:WORKHORSE_BENCHMARK_03__0_0}
    \end{subfigure}

    \begin{subfigure}[b]{\textwidth}
        \centering
        \includegraphics[width=0.35\textwidth]{./fragments/05_workhorse_experiment/images/01_basebenchmark_03_sort_d_case.png.1_0.png}
        \includegraphics[width=0.35\textwidth]{./fragments/05_workhorse_experiment/images/01_basebenchmark_03_sort_d_case.png.1_1.png}
        \caption{Extraction time}
        \label{FIG:WORKHORSE_BENCHMARK_03__0_1}
    \end{subfigure}

    \begin{subfigure}[b]{\textwidth}
        \centering
        \includegraphics[width=0.35\textwidth]{./fragments/05_workhorse_experiment/images/01_basebenchmark_03_sort_d_case.png.2_0.png}
        \includegraphics[width=0.35\textwidth]{./fragments/05_workhorse_experiment/images/01_basebenchmark_03_sort_d_case.png.2_1.png}
        \caption{Stack size}
        \label{FIG:WORKHORSE_BENCHMARK_03__0_2}
    \end{subfigure}
    
    \caption{bIIQS benchmark for a descending sorted sequence with $100\times10^3$ elements.}
       \label{FIG:WORKHORSE_BENCHMARK_03}
\end{figure}


\subsection{Performance comparison for repeated sequences}

For single class instances, which represent the worst case for IIQS when dealing with repeated sequences, bIIQS shows more than three orders of magnitude boost in performance in comparison to IIQS. As shown in Figure~\ref{FIG:WORKHORSE_BENCHMARK_05}, and ~\ref{FIG:WORKHORSE_BENCHMARK_05_NOISE}, while our single class input continues to represent our worst case, it does not pose a problem as before. \\

Experimental results shown in Figure~\ref{FIG:WORKHORSE_BENCHMARK_04} reveal that bIIQS does not only perform the fastest first element extraction, but also that formerly stated properties of IIQS as optimal space usage and fastest extractions are also preserved. As the stack now stores a range of elements, each extraction performed on it does not require any extra partition stage as it is being immediately delivered. This makes the ---optional--- pop-push operation the only overhead in this implementation.\\

\begin{figure}
    %BENCHMARK_05_CLASSES
    %01_basebenchmark_05_classes
    \centering
    \begin{subfigure}[b]{\textwidth}
        \centering
        \includegraphics[width=0.35\textwidth]{./fragments/05_workhorse_experiment/images/01_basebenchmark_05_classes.png.1_0.png}
        \includegraphics[width=0.35\textwidth]{./fragments/05_workhorse_experiment/images/01_basebenchmark_05_classes.png.1_1.png}
        \caption{IQS}
        \label{FIG:WORKHORSE_BENCHMARK_05_NOISE__0_0}
    \end{subfigure}

    \begin{subfigure}[b]{\textwidth}
        \centering
        \includegraphics[width=0.35\textwidth]{./fragments/05_workhorse_experiment/images/01_basebenchmark_05_classes.png.0_1.png}
        \includegraphics[width=0.35\textwidth]{./fragments/05_workhorse_experiment/images/01_basebenchmark_05_classes.png.0_2.png}
        \caption{IIQS}
        \label{FIG:WORKHORSE_BENCHMARK_05_NOISE__0_1}
    \end{subfigure}

    \caption{bIIQS benchmark for class impact on extraction time.}
    \label{FIG:WORKHORSE_BENCHMARK_05_NOISE}
\end{figure}


\begin{figure}
    %BENCHMARK_04_SINGLE_CLASS
    %01_basebenchmark_04_single_class
    \centering
    \begin{subfigure}[b]{\textwidth}
        \centering
        \includegraphics[width=0.35\textwidth]{./fragments/05_workhorse_experiment/images/01_basebenchmark_04_single_class.png.0_0.png}
        \includegraphics[width=0.35\textwidth]{./fragments/05_workhorse_experiment/images/01_basebenchmark_04_single_class.png.0_1.png}
        \caption{Cumulated extraction time}
        \label{FIG:WORKHORSE_BENCHMARK_04__0_0}
    \end{subfigure}

    \begin{subfigure}[b]{\textwidth}
        \centering
        \includegraphics[width=0.35\textwidth]{./fragments/05_workhorse_experiment/images/01_basebenchmark_04_single_class.png.1_0.png}
        \includegraphics[width=0.35\textwidth]{./fragments/05_workhorse_experiment/images/01_basebenchmark_04_single_class.png.1_1.png}
        \caption{Extraction time}
        \label{FIG:WORKHORSE_BENCHMARK_04__0_1}
    \end{subfigure}

    \begin{subfigure}[b]{\textwidth}
        \centering
        \includegraphics[width=0.35\textwidth]{./fragments/05_workhorse_experiment/images/01_basebenchmark_04_single_class.png.2_0.png}
        \includegraphics[width=0.35\textwidth]{./fragments/05_workhorse_experiment/images/01_basebenchmark_04_single_class.png.2_1.png}
        \caption{Stack size}
        \label{FIG:WORKHORSE_BENCHMARK_04__0_2}
    \end{subfigure}
    
    \caption{bIIQS benchmark for a random sorted sequence with $100\times10^3$ repeated elements.}
    \label{FIG:WORKHORSE_BENCHMARK_04}
\end{figure}


As for the noise impact we can clearly identify that now the bias on the resulting index provided by the three-way partition does not impact on the performance of bIIQS except for very small values (see Figure~\ref{FIG:WORKHORSE_BENCHMARK_05}). In contrast, the noise now has a slight negative impact on bIIQS which can be attributed to the extra step performed to stack a pair of elements instead of a single one. Even with this degradation, the performance on all cases is stable in comparison to IIQS and delivers results in the expected running time, on pair with random sequences of unique elements.\\


\begin{figure}
    %WORKHORSE_BENCHMARK_05
    %01_basebenchmark_06_noise_redundant_bias
    \centering
    \begin{subfigure}[b]{0.45\textwidth}
        \centering
        \includegraphics[width=0.9\textwidth]{./fragments/05_workhorse_experiment/images/01_basebenchmark_06_noise_redundant_bias.png.1_0.png}
        % \includegraphics[width=0.35\textwidth]{./fragments/05_workhorse_experiment/images/01_basebenchmark_06_noise_redundant_bias.png.1_1.png}
        % \includegraphics[width=0.35\textwidth]{./fragments/05_workhorse_experiment/images/01_basebenchmark_06_noise_redundant_bias.png.1_2.png}
        \caption{IIQS}
        \label{FIG:WORKHORSE_BENCHMARK_05__0_0}
    \end{subfigure}
    \hfill
    \begin{subfigure}[b]{0.45\textwidth}
        \centering
        \includegraphics[width=0.9\textwidth]{./fragments/05_workhorse_experiment/images/01_basebenchmark_06_noise_redundant_bias.png.0_0.png}
        %\includegraphics[width=0.35\textwidth]{./fragments/05_workhorse_experiment/images/01_basebenchmark_06_noise_redundant_bias.png.0_1.png}
        \caption{bIIQS}
        \label{FIG:WORKHORSE_BENCHMARK_05__0_1}
    \end{subfigure}

    \caption{Random noise and pivot bias benchmark for bIIQS impact on extraction time.}
    \label{FIG:WORKHORSE_BENCHMARK_05}
\end{figure}

We confirm our hypothesis by examining both noise impact and three-way-partition bias as separate elements depicted in Figures~\ref{FIG:WORKHORSE_BENCHMARK_06} and~\ref{FIG:WORKHORSE_BENCHMARK_07} respectively.\\


\begin{figure}
    %WORKHORSE_BENCHMARK_06
    %01_basebenchmark_06_redundant_bias
    \centering
    \begin{subfigure}[b]{0.45\textwidth}
        \centering
        \includegraphics[width=0.9\textwidth]{./fragments/05_workhorse_experiment/images/01_basebenchmark_06_redundant_bias.png.1_0.png}
        % \includegraphics[width=0.35\textwidth]{./fragments/05_workhorse_experiment/images/01_basebenchmark_06_redundant_bias.png.1_1.png}
        % \includegraphics[width=0.35\textwidth]{./fragments/05_workhorse_experiment/images/01_basebenchmark_06_redundant_bias.png.1_2.png}
        \caption{IIQS}
        \label{FIG:WORKHORSE_BENCHMARK_06__0_0}
    \end{subfigure}
    \hfill
    \begin{subfigure}[b]{0.45\textwidth}
        \centering
        \includegraphics[width=0.9\textwidth]{./fragments/05_workhorse_experiment/images/01_basebenchmark_06_redundant_bias.png.0_0.png}
        %\includegraphics[width=0.35\textwidth]{./fragments/05_workhorse_experiment/images/01_basebenchmark_06_redundant_bias.png.0_1.png}
        \caption{bIIQS}
        \label{FIG:WORKHORSE_BENCHMARK_06__0_1}
    \end{subfigure}

    \caption{Pivot bias benchmark for bIIQS impact on extraction time.}
    \label{FIG:WORKHORSE_BENCHMARK_06}
\end{figure}


\begin{figure}
    %WORKHORSE_BENCHMARK_07
    %01_basebenchmark_06_noise_bias
    \centering
    \begin{subfigure}[b]{0.45\textwidth}
        \centering
        \includegraphics[width=0.9\textwidth]{./fragments/05_workhorse_experiment/images/01_basebenchmark_06_noise_bias.png.1_0.png}
        % \includegraphics[width=0.35\textwidth]{./fragments/05_workhorse_experiment/images/01_basebenchmark_06_noise_bias.png.1_1.png}
        % \includegraphics[width=0.35\textwidth]{./fragments/05_workhorse_experiment/images/01_basebenchmark_06_noise_bias.png.1_2.png}
        \caption{IIQS}
        \label{FIG:WORKHORSE_BENCHMARK_07__0_0}
    \end{subfigure}
    \hfill
    \begin{subfigure}[b]{0.45\textwidth}
        \centering
        \includegraphics[width=0.9\textwidth]{./fragments/05_workhorse_experiment/images/01_basebenchmark_06_noise_bias.png.0_0.png}
        %\includegraphics[width=0.35\textwidth]{./fragments/05_workhorse_experiment/images/01_basebenchmark_06_noise_bias.png.0_1.png}
        \caption{bIIQS}
        \label{FIG:WORKHORSE_BENCHMARK_07__0_1}
    \end{subfigure}

    \caption{Random noise impact on bIIQS extraction time.}
    \label{FIG:WORKHORSE_BENCHMARK_07}
\end{figure}


