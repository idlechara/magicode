%% This should be the last thing to write

% \chapterimage[height=3cm]{./fragments/memes/internet_archive}

\section{Context}

Since early computing, \emph{the sorting problem} has been a focus of interest of many researchers. Challenges range from efficient implementations for certain machines or architectures, minimizing asymptotic complexity, to handling special cases or structures. Not surprisingly, most programs ---if not all of them--- rely on a sorting process as part of their procedures.\\

In algorithm design, pure theoretical analyses are the preferred way to prove both correctness and limits of algorithms. Usually these analyses are performed by using a series of mathematical techniques, which aim to provide generality of the results. As such, these analysis are usually developed to establish space and time complexities which are later used to determine best use cases and feasibility of implementation.\\

But in practice, when it comes to implementing these algorithms, there are many factors that can hinder the development process. The same algorithm can perform better depending on the distribution of the input, on which case this becomes the subject of our analysis. On others, computer architecture has a huge impact on the running time. Even initial values for a variable ---regardless if it is being assigned during compile time or run time--- can induce different results which usually lead into false and misleading outputs. \\

In order to close the gap generated between theory and practice, the concept of \emph{Experimental Algoritmics} is born. Extrapolating \emph{Design of Experiments} practices to algorithm design and engineering, this approach targets to ease the development of new programs by taking a ``nuts and bolts'' approach complementing theoretical analysis with practical results. \\

There are many algorithms developed using this process, and one of those is an incremental sorting algorithm known as \emph{Introspective Incremental Quick Sort}. This is a extension of an incremental sorting algorithm called \emph{Incremental Quick Sort}, intended to overcome their worst-case scenario by using the concept of \emph{algorithm introspection} used on \emph{IntroSort}\cite{MusserIntroSort}.\\

\section{Application areas}

In contrast to the most common sorting algorithms like \emph{QuickSort}, incremental sorting algorithms are designed to retrieve elements from a sequence in an ordered fashion, without sorting the entire array. This kind of behaviour is exploited on structures like the optimal minimum binary heaps used to implement priority queues. \\

Thus, as a continuous extraction of all elements via incremental sorting ends with a completely sorted dataset, incremental sorting is by extension a \emph{resumeable} sorting.

This is useful in process which rely on heavy sorting processes that demand constant updates to the user. This is the example of \emph{Haplotype Plots} generation in programs like \emph{SLiM}~\cite{msw211}, on which incremental sorting provides a way to constantly inform the user in cases on which the sorting process can take hours to complete.\\

\section{Problem description}
Both Incremental Quick Sort and Introspective Incremental Quick Sort as defined in their original papers \cite{7416566} do not enforce any particular subroutines to be used to fill in the gaps of the procedures to follow. As such, the behaviour of both algorithms are subjected to change in function on which subroutines are used to implement it as an actual program. This leads to problems like being unable to sort sequences when repeated elements are present, due to the implementation of the partition subroutine.\\

The original work for Introspective Incremental QuickSort (IIQS) \cite{7416566} contains a rough theoretical analysis for its worst-case performance, and a rather simple execution analysis used it to compare it with other state-of-the-art algorithms. While most of the time this information is enough to introduce an algorithm and validate their theoretical constraints, developers may struggle as they try to reproduce the reference implementation and test them in order to make use of them into actual applications\footnote{For Introspective Incremental Quick Sort, there is a reference implementation available on one of the author's GitHub page, but in order to comply with the conference standards at the moment of publishing, such data were not included in the paper but mentioned its presentation.}.\\ 

\section{Goals}
The objective of this document is to compile the efforts behind the use of an experimental algorithmics approach to develop an improved version of Introspective Incremental Quick Sort. This implementation aims to maintain its properties for the standard case of sorting sequences with no repeated elements, but it is also capable of managing repeated elements without changing its complexity.\\

\subsection{General goals}
The main objective in this work is to \emph{``use an experimental algorithmics approach to extend Introspective Incremental Quick Sort's implementation to accept repeated sequences as input''}.\\

\subsection{Specific goals}
In order to reach our specific goal, we need to complete the following tasks:

\begin{itemize}
    \item Research on the foundations of Incremental Quick Sort and Introspective Incremental Quick Sort.
    \item Plan pilot experiments to gain insight on potential improvements.
    \item Apply this insight to design a new version of the target algorithm.
    \item Empirically prove its correctness by benchmarking it. 
\end{itemize}

\section{Document Structure}
The document is organized in the same way as the design process has been followed, from the initial background research, each experiment performed and final implementation.\\

\emph{Chapter~\ref{CHAPTER:BACKGROUND}:} This chapter is focused on explaining all relevant details in order to understand the problem at hand. It presents an introduction to our base algorithms, the methodology used, tools and techniques needed for the reader to understand the next chapters.\\

\emph{Chapter~\ref{CHAPTER:METHODLOGY}:} This chapter explains how the methodology introduced on the previous section is applied to this work. Details as experimental setup and implementation concerns,
measuring criteria and a breakdown of the process is presented here.\\

\emph{Chapter~\ref{CHAPTER:EXPERIMENTAL_EXECUTION}:} This chapter presents a series of pilots experiments used to understand more the problem and to guide our algorithm engineering process. Each experiment covers their own additional background as research is part of this process.\\

\emph{Chapter~\ref{CHAPTER:WORKHORSE_EXPERIMENT}:} This chapter presents our algorithm proposal and the experiments used to prove its correctness.\\

\emph{Chapter~\ref{CHAPTER:SUMMARY}:} This chapter is focused on discussing the results of this process and to expose our findings.